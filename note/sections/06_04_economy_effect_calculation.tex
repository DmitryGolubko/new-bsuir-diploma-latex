\subsection{Расчет экономического эффекта}
\label{sec:economics:economy_effect_calculation}

В процессе использования нового ПС чистая прибыль в конечном итоге возмещает капитальные затраты. Однако, полученные
при этом суммы результатов (прибыли) и затрат (капитальных вложений) по годам приводят к единому времени -- расчетному
году (за расчетный год принят год разработки ДП) путем умножения результатов и затрат за каждый год на коэффициент
привидения ($\text{ALFA}_\text{t}$), который рассчитывается по формуле:

\begin{equation}
  \text{ALFA}_\text{t} = (1 + \text{E}_\text{н})^{\text{t}_\text{р}-\text{t}},
\end{equation}
\begin{explanation}
  где & $ \text{E}_\text{н} $ & норматив привидения разновременных затрат и результатов;\\
  & $ \text{t}_\text{р} $ & расчетный год, $ \text{t}_\text{р} = 1$; \\
  & $ \text{t} $ & номер года, результаты и затраты которого приводятся к расчетному.
\end{explanation}

\begin{equation*}
  \text{ALFA}_\text{1} = (1 + 0,2)^{1-1} = 1;
\end{equation*}
\begin{equation*}
  \text{ALFA}_\text{2} = (1 + 0,2)^{1-2} = 0,833;
\end{equation*}
\begin{equation*}
  \text{ALFA}_\text{3} = (1 + 0,2)^{1-3} = 0,690;
\end{equation*}
\begin{equation*}
  \text{ALFA}_\text{4} = (1 + 0,2)^{1-4} = 0,579.
\end{equation*}

Данные расчета экономического эффекта целесообразно свести в
таблицу~\ref{table:economics:economy_effect_calculation:effect}.

\begin{longtable}{|>{\centering}m{0.17\textwidth}
  |p{0.08\textwidth}
  |p{0.15\textwidth}
  |p{0.12\textwidth}
  |p{0.10\textwidth}
  |p{0.10\textwidth}
  |>{\centering\arraybackslash}m{0.10\textwidth}|}
\caption{Расчет экономического эффекта от использования нового ПС}
\label{table:economics:economy_effect_calculation:effect}\\

\hline
  \centering Показатели & \centering Ед. измерения & \centering Методика расчета & \centering 2019 & \centering 2020 & \centering 2021 & \centering\arraybackslash 2022 \endfirsthead

\caption*{Продолжение таблицы \ref{table:economics:economy_effect_calculation:effect}}\\\hline
\centering Показатели & \centering Ед. измерения & \centering Методика расчета & \centering 2019 & \centering 2020 & \centering 2021 & \centering\arraybackslash 2022 \\\hline \endhead

\hline
	Прирост прибыли за счет экономии затрат & \centering руб. & \centering $\Delta\text{П}_\text{ч}$ & \centering 4356,24 & \centering 4356,24 & \centering 4356,24 & 4356,24 \\

  Сумма прибыли с учетом фактора времени & \centering руб. & \centering $\Delta\text{П}_\text{ч} \cdot \text{ALFA}_\text{t}$ & \centering 4356,24 & \centering 3628,75 & \centering 3005,81 & 2522,26 \\
\hline
  Сумма затрат & \centering руб. & \centering $\text{К}_\text{о}$ & \centering 7914,91 & \centering --- & \centering --- & --- \\
\hline
  Сумма затрат с учетом фактора времени & \centering руб. & \centering $\text{К}_\text{о} \cdot \text{ALFA}_\text{t}$ & \centering 7914,91 & \centering --- & \centering --- & --- \\
\hline
  Эконом. эффект & \centering руб. & \centering $\Delta\text{П}_\text{ч} \cdot  \cdot \text{ALFA}_\text{t} - -\text{К}_\text{о} \cdot \cdot\text{ALFA}_\text{t}$ & \centering --3558,67 & \centering 3628,75 & \centering 3005,81 & 2522,26 \\
\hline
  Эконом. эффект с нарастающим  итогом & \centering руб. &  & \centering --3558,67 & \centering 70,08 & \centering 3075,89 & 5598,15 \\
\hline
  Коэффициент приведения & \centering ед. & \centering $\text{ALFA}_\text{t}$ & \centering 1 & \centering 0,833 & \centering 0,690 & 0,579 \\
\hline
\end{longtable}

Таким образом, все затраты на разработку системы менеджмента персонала полностью окупятся на второй год.

Рентабельность инвестиций в разработку и внедрение программного средства по следующей формуле:
\begin{equation}
  \text{Р}_\text{и} = \frac{\text{П}_\text{чср}}{\text{К}_\text{пр}} \cdot 100\%,
\end{equation}
\begin{explanation}
  где & $ \text{П}_\text{чср} $ & среднегодовая величина чистой прибыли за расчетный период, руб.
\end{explanation}
\pagebreak
\begin{equation*}
  \text{Р}_\text{и} = \frac{\frac{\sum_{i=1}^{4} \Delta\text{П}_\text{чi}}{4}}{\text{К}_\text{o}} \cdot 100\% = \frac{3378,27}{7914,91} \cdot 100\% = 42\%.
\end{equation*}
\setlength{\parskip}{0pt}

В результате технико-экономического обоснования применения программного продукта были получены следующие показателей:
\begin{itemize}
	\item чистый дисконтированный доход за четыре года составит $5598,15$ руб.;
	\item затраты на разработку программного продукта окупятся на второй год использования;
	\item рентабельность инвестиций составляет $42\%$.
\end{itemize}

Таким образом, применение программного продукта является эффективным и инвестиции в его разработку целесообразно
осуществлять.
