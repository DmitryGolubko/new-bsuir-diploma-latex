\subsection{Расчет затрат на разработку программного обеспечения}
\label{sec:economics:cost_calculation}

В соответствии с «Рекомендациями по применению «Единой тарифной сетки» рабочих и служащих народного хозяйства» и
тарифными разрядами и коэффициентами должностей каждому исполнителю устанавливается разряд и тарифный коэффициент.

Месячная тарифная ставка каждого исполнителя определяется путем умножения действующей месячной тарифной ставки 1-го
разряда на тарифный коэффициент, соответствующий установленному тарифному разряду и рассчитывается по формуле:
\pagebreak

\begin{equation}
	\text{З}_\text{зм} = \text{З}_\text{зм}^1 \cdot \text{K}_\text{т},
\end{equation}
\begin{explanation}
  где & $ \text{З}_\text{зм} $ & тарифная ставка за месяц, руб.;\\
  & $ \text{З}_\text{зм}^1 $ & тарифная ставка 1-го разряда за месяц, руб.;\\
  & $ \text{K}_\text{т} $ & тарифный коэффициент, ед.
\end{explanation}
\begin{equation}
  \text{З}_\text{зм} = 290,97 \cdot 2,74 = 797,26 \text{ руб.},
\end{equation}

Основная заработная плата исполнителей на конкретное ПС рассчитывается по формуле:

\begin{equation}
	\text{З}_\text{оз} = \text{З}_\text{зд} \cdot \text{Т}_\text{о} \cdot \text{К}_\text{п} \cdot \text{К}_\text{пр},
\end{equation}
\begin{explanation}
  где & $ \text{З}_\text{оз} $ & основная заработная плата, руб.;\\
  & $ \text{З}_\text{зд} $ & тарифная ставка за день ($ \text{З}_\text{зм} $ разделить на $21,25$), руб.;\\
  & $ \text{Т}_\text{о} $ & общая трудоемкость ПС, человеко-дней;\\
  & $ \text{К}_\text{п} $ & коэффициент естественных потерь рабочего времени, ед.;\\
  & $ \text{К}_\text{пр} $ & коэффициент премирования, $ \text{К}_\text{пр} = 1,2 $;
\end{explanation}

\begin{equation}
  \text{З}_\text{зд} = \frac{797,26}{21,25}  = 37,52 \text{ руб.},
\end{equation}

\begin{equation}
	\text{З}_\text{оз} = 37,52 \cdot 39,62 \cdot 1,3 \cdot 1,2 = 2319 \text{ руб.},
\end{equation}

Дополнительная заработная плата на конкретное ПС включает выплаты, предусмотренные законодательством о труде
(оплата отпусков, льготных часов, времени выполнения государственных обязанностей и других выплат, не связанных с
основной деятельностью исполнителей), и определяется по нормативу в процентах к основной заработной плате:

\begin{equation}
  \text{З}_\text{дз} = \frac{\text{З}_\text{оз} \cdot \text{Н}_\text{дз} }{100},
\end{equation}
\begin{explanation}
  где & $ \text{З}_\text{дз} $ & дополнительная заработная плата на конкретное ПС, руб.;\\
  & $ \text{Н}_\text{дз} $ & норматив дополнительной заработной платы, ($ 22\% $);
\end{explanation}

\begin{equation}
  \text{З}_\text{дз} = \frac{ 2319 \cdot 22}{100}  = 510,18 \text{ руб.},
\end{equation}

Отчисления на социальные нужды включают в предусмотренные законодательством отчисления в фонд социальной защиты ($ 34\% $)
и фонд обязательного страхования ($ 0,6\% $) в процентах от основной и дополнительной заработной платы и рассчитываются
по формуле:

\begin{equation}
  \text{З}_\text{соц} = \frac{(\text{З}_\text{оз} + \text{З}_\text{дз}) \cdot \text{Н}_\text{соц} }{100}
\end{equation}

\begin{equation}
  \text{З}_\text{соц} = \frac{(2319 + 510,18) \cdot 34,6 }{100} = 978,90 \text{ руб.}
\end{equation}

Расходы по статье «Машинное время» ($ \text{Р}_\text{мв} $) включают оплату машинного времени, необходимого для разработки и отладки ПС, и
определяются по формуле:

\begin{equation}
  \text{Р}_\text{мв} = \text{Ц}_\text{м} \cdot \text{Т}_\text{ч} \cdot \text{С}_\text{р},
\end{equation}
\begin{explanation}
  где & $ \text{Ц}_\text{м} $ & цена одного машино-часа;\\
  & $ \text{Т}_\text{ч} $ & количество часов работы в день;\\
  & $ \text{С}_\text{р} $ & длительность проекта.
\end{explanation}

Стоимость машино-часа на предприятии составляет 1,5 руб. Разработка проекта займет 40 дней. Определим затраты по
статье «Машинное время»:

\begin{equation}
  \text{Р}_\text{мв} = 1,5 \cdot 8 \cdot 40 = 480 \text{ руб.}
\end{equation}

Расчет прочих затрат осуществляется в процентах от затрат на основную заработную плату разработчиков с учетом
премии по формуле:

\begin{equation}
  \text{З}_\text{пз} = \frac{\text{З}_\text{оз} \cdot \text{Н}_\text{пз}}{100},
\end{equation}
\begin{explanation}
  где & $ \text{Н}_\text{пз} $ & норматив прочих затрат, берется в пределах, равный $100\%$ от основной заработной платы.
\end{explanation}

\begin{equation}
  \text{З}_\text{пз} = \frac{2319 \cdot 100}{100} = 2319 \text{ руб.}
\end{equation}

Затраты по статье «Накладные расходы» определяются по формуле:

\begin{equation}
  \text{Р}_\text{н} = \frac{\text{З}_\text{оз} \cdot \text{Н}_\text{рн}}{100},
\end{equation}
\begin{explanation}
  где & $ \text{Н}_\text{рн} $ & процент накладных расходов, ($55\%$).
\end{explanation}

\begin{equation}
  \text{Р}_\text{н} = \frac{2319 \cdot 55}{100} = 1275,45 \text{ руб.}
\end{equation}

Общая сумма расходов по смете на ПО составит:

\begin{equation}
	\text{С}_\text{р} = \text{З}_\text{оз} + \text{З}_\text{дз} + \text{З}_\text{соц} + \text{Р}_\text{мв} + \text{З}_\text{пз} + \text{Р}_\text{н},
\end{equation}

\begin{equation}
	\text{С}_\text{р} = 2319 + 510,18 + 978,90 + 480 + 2319 + 1275,45 = 6882,53 \text{ руб.}
\end{equation}

Полная сумма затрат на разработку программного обеспечения приведена в таблице~\ref{table:economics:cost_calculation:total_cost}.

\begin{table}[!ht]
  \caption{Исходные данные}
  \label{table:economics:cost_calculation:total_cost}
  \centering
    \begin{tabular}{{
    |>{\raggedright}m{0.75\textwidth} | 
    >{\centering\arraybackslash}m{0.2\textwidth}|}}
      \hline
        {\begin{center} Статья затрат \end{center}} & {\begin{center} Сумма, руб \end{center}}\\
      \hline
        Основная заработная плата команды разработчиков & 2319 \\
      \hline
        Дополнительная заработная плата команды разработчиков & 510,18 \\
      \hline
        Затраты на машинное время & 480 \\
      \hline
        Отчисления на социальные нужды & 978,90 \\
      \hline
        Прочие затраты & 2319 \\
      \hline
        Накладные расходы  & 1275,45 \\
      \hline
        Общая сумма затрат на разработку & 6882,53 \\
      \hline
    \end{tabular}
\end{table}

Расходы на освоение разработчиком ПО рассчитываются по формуле:

\begin{equation}
  \text{Р}_\text{о} = \frac{\text{С}_\text{р} \cdot \text{Н}_\text{о}}{100},
\end{equation}
\begin{explanation}
  где & $ \text{Н}_\text{о} $ & норматив расходов на освоение, $15\%$.
\end{explanation}

\begin{equation}
  \text{З}_\text{пз} = \frac{6882,53 \cdot 15}{100} = 1032,38 \text{ руб.}
\end{equation}

Полная себестоимость разработанного ПО определяется по следующей формуле:

\begin{equation}
  \text{С}_\text{п} = \text{С}_\text{р} + \text{Р}_\text{о},
\end{equation}

\begin{equation}
  \text{С}_\text{п} = 6882,53 + 1032,38 = 7914,91 \text{ руб.}
\end{equation}

Таким образом, общие затраты предприятия на разработку системы управления бизнес-процессами составят $7914,91$ руб.
