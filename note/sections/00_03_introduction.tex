\setcounter{page}{2}
\sectioncentered*{Краткое Введение}


Переход к рыночным отношениям в экономике и научно-технический прогресс чрезвычайно ускорили темпы внедрения во все
сферы социально-экономической жизни общества последних научных разработок в области информационных технологий.

Рынок недвижимости представляет собой механизм, обслуживающий и регулирующий отношения по купле,
продаже и аренде недвижимости на основе спроса и предложения.
В мировой практике можно выделить следующие типы рынков недвижимости:
\begin{itemize}
	\item рынок жилой недвижимости;
	\item рынок коммерческой недвижимости, приносящей доход ее владельцу (офисные, торговые, производственные,
  складские помещения);
  \item рынок земельных участков.
\end{itemize}

Рынок недвижимости делится на первичный и вторичный. Объектом сделок на первичном рынке является
новая недвижимость, т.е. только что построенные дома, квартиры, офисные и другие помещения. Их могут продавать
застройщики, инвесторы, финансировавшие строительство. На вторичном рынке предоставлено жилье и помещения,
которыми уже пользовались по основному назначению. Первичный рынок отражает объемы созданной жилой недвижимости, а
объем вторичного рынка определяется другими факторами:
\begin{itemize}
  \item изменением благосостояния населения;
  \item доходностью различных инвестиционных объектов;
  \item мобильностью трудовых ресурсов;
  \item событиями человеческой жизни (свадьба, развод, рождение ребенка в семье, смена места жительства и др.).
\end{itemize}

Объекты недвижимости занимают значительную часть ресурсов экономики любой страны.
Оценка стоимости – длительный и сложный процесс установления денежного эквивалента стоимости объекта недвижимости.
Она требует высокой квалификации оценщика, владеющего методами и инструментарием оценочной деятельности, знающего состояние
рынка недвижимости и особенно нужного сегмента, детального значения правовых особенностей сделок с недвижимостью и др.
Практика показывает, что для оценки стоимости объекта недвижимости, специалисту требуется значительное время.

С развитием теоретических подходов для создания адекватных моделей поведения рынка недвижимости в западных странах и
США одновременно происходило активное внедрение новых интеллектуальных компьютерных технологий в практику принятия
финансовых и инвестиционных решений. Вначале в виде экспертных систем и баз знаний, а затем с конца 80-х - нейросетевых
технологий, которые являются адекватным аппаратом для решения задач прогнозирования.

Начало исследования методов обработки информации, называемых сегодня нейросетевыми, было положено несколько десятилетий
назад. С течением времени интерес к нейросетевым технологиям то ослабевал, то вновь возрождался. Такое непостоянство
напрямую связано с практическими результатами проводимых исследований.

Цена на жилье - вопрос крайне сложный, изучение основных факторы влияния и выяснение правил изменения имеют важное теоретическое и
практическое значение для способствования устойчивому и здоровому развитию рынка жилой недвижимости.
Автоматизация позволит ускорить процесс принятия решения, учесть
большее количество факторов и снизить уровень субъективности.

\begin{center}
  \textbf{ОБЩАЯ ХАРАКТЕРИСТИКА РАБОТЫ}\\
\end{center}

Цель диссертационной работы – провести анализ применяемых моделей и алгоритмов, которые используются при
прогнозировании цен на рынке недвижимости, проверить адекватность применения методов эконометрического
анализа для оценки объектов недвижимости и построение на их основе модели стоимости, применить существующие алгоритмы на
подготовленной выборке данных.
На основе проведенного анализа выполнить сравнение алгоритмов, определить наиболее эффективный из них.
Результаты исследования могут быть полезны для прогнозирования
ценообразования на рынке недвижимости, а также при оценке стоимости объектов недвижимости.

Для достижения поставленной цели необходимо решить следующие задачи:
\begin{enumerate}
  \item Провести анализ применяемых моделей и алгоритмов, которые используются при прогнозировании цен на рынке недвижимости.
  \item Реализовать предложенные модели и провести экспериментальные исследования на основе накопленных данных.
  \item На основе сравнительного анализа выбрать более эффективный и достоверный.
\end{enumerate}

\textit{Объектом} исследования выступает процесс формирования цен \linebreak на недвижимость.

\textit{Предметом} исследования являются модели и алгоритмы, которые используются для анализа процесса формирования цен.

Основной \textit{гипотезой}, положенной в основу диссертационной работы, является сравнение результатов работы выбранных алгоритмов
и выявление более эффективного способа оценки стоимости недвижимости.
~\\
% \textbf{Опубликованность результатов диссертации}\\
% По теме диссертации опубликовано 2 печатных работы в сборниках материалов международных научных конференций.

\textbf{Связь работы с приоритетными направлениями научных исследований и запросами реального сектора экономики}\\

Работа выполнялась в соответствии с научно-техническим заданием и планом работ кафедры «Программное обеспечение
информационных технологий» по теме «Разработка моделей, методов, алгоритмов, повышающих показатели проектирования,
внедрения и эксплуатации программных средств для перспективных платформ обработки информации, решения интеллектуальных
задач, работы с большими массивами данных и внедрение в современные обучающие комплексы» (ГБ  № 16-2004, № ГР 20163588,
научный руководитель НИР -- Н. В. Лапицкая).
~\\

\textbf{Личный вклад соискателя}\\

Результаты, приведенные в диссертации, получены соискателем лично. Вклад научного руководителя О. Г. Смоляковой
заключается в формулировке целей и задач исследования.
~\\

\textbf{Опубликованность результатов диссертации}\\

По теме диссертации опубликовано 2 печатных работы, одна работа -- в сборнике материалов IХ Республиканской
научно-практической конференции «Вычислительные методы, модели и образовательные технологии» 2020~г, вторая -- в
научном журнале «Студенческий форум».
~\\

\textbf{Структура и объем диссертации}\\

Диссертация состоит из введения, общей характеристики работы, двух глав, заключения, списка использованных источников,
списка публикаций автора и приложений. В первой главе представлен анализ cуществующих применяемых моделей и алгоритмов.
Вторая глава посвящена проведению эксперименального исследования, реализацией выбранных алгоритмов анализа, проведению
их сравнительного анализа.

Общий объем работы составляет 57 страниц, из которых основного текста -- 45 страниц, 27 рисунков на 23 страницах,
1 таблица на 1 странице, список использованных источников из 12 наименований на 1 странице и 1 приложение на 9 страницах.

\begin{center}
  \textbf{ОСНОВНОЕ СОДЕРЖАНИЕ}\\
\end{center}

Во \textbf{введении} определена область и указаны основные направления исследования, показана актуальность темы
диссертационной работы, дана краткая характеристика исследуемых вопросов, обозначена практическая ценность работы.

В \textbf{первой главе} проведен анализ существующих алгоритмов и моделей, выбраны наиболее подходящие для реализации
в эксперементальной части. Приведены и проанализированы существующие аналоги.

В \textbf{второй главе} показана подготовка выборки для реализации существующих моделей и алгоритмов, 
непосредственно сама реализация, были проанализированы полученных данных и выполнено сравнение эффективности реализованных
алгоритмов, сделаны соответствующие выводы о возможности применения результатов на практике.

\begin{center}
  \textbf{ЗАКЛЮЧЕНИЕ}\\
\end{center}

В ходе работы над магистерской диссертацией были изучены и проанализированы существующие способы и алгоритмы анализа
данных. Были собраны данные об о более чем 5 тысяч проданных объектах недвижимости на рынке за период с 2017 по 2020 год.
После был выполнен регресионный анализ
полученных данных и анализ с использованием нейронных сетей. Было проведено сравнение полученных результатов и сделаны
соответствующие выводы об их эффективности и возможности реального использования.

На основании проведенных исследований можно утверждать, что применение нейронных сетей для прогнозирования
стоимости объектов недвижимости более эффективно использования методов регрессионного анализа и может достаточно
точно отражать рыночную стоимость недвижимости. Однако с увеличением числа выборки точность прогнозирования падает ввиду
присутствия множества скрытых факторов(такие как время постройки дома, удаленность от различных сервисов, наличие или
отсутствие мебели, качество постройки дома), не учитываемых в данном исследовании.
Предложенные методы могут быть использованы продавцами для первичной оценки стоимости жилой недвижимости,
а покупателями могут использоваться в качестве дополнительного источника информации, однако данная модель требует
доработки.

\sectioncentered*{Список опубликованных работ}

[1] Д.~В.~Голубко. Анализ цен на рынке недвижимости~/~Д.~В.~Голубко~//~IХ Республиканская научно-практическая
конференция «Вычислительные методы, модели и образовательные технологии»: сборник материалов. – Брест: БрГУ, 2020.

[2] Д.~В.~Голубко. Анализ цен на рынке недвижимости~/~Д.~В.~Голубко~//~Студенческий форум: электрон. научн. журн. 2021. № 2(138).\linebreak
URL: https://nauchforum.ru/journal/stud/138/84832.
\newpage

