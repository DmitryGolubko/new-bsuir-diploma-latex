
\sectioncentered*{Введение}
\addcontentsline{toc}{section}{ВВЕДЕНИЕ}
\label{sec:introduction}



Переход к рыночным отношениям в экономике и научно-технический прогресс чрезвычайно ускорили темпы внедрения во все
сферы социально-экономической жизни общества последних научных разработок в области информационных технологий.

Рынок недвижимости представляет собой механизм, обслуживающий и регулирующий отношения по купле,
продаже и аренде недвижимости на основе спроса и предложения.
В мировой практике можно выделить следующие типы рынков недвижимости:
\begin{itemize}
	\item рынок жилой недвижимости;
	\item рынок коммерческой недвижимости, приносящей доход ее владельцу (офисные, торговые, производственные,
  складские помещения);
  \item рынок земельных участков.
\end{itemize}

Рынок недвижимости делится на первичный и вторичный. Объектом сделок на первичном рынке является
новая недвижимость, т.е. только что построенные дома, квартиры, офисные и другие помещения. Их могут продавать
застройщики, инвесторы, финансировавшие строительство. На вторичном рынке предоставлено жилье и помещения,
которыми уже пользовались по основному назначению. Первичный рынок отражает объемы созданной жилой недвижимости, а
объем вторичного рынка определяется другими факторами:
\begin{itemize}
  \item изменением благосостояния населения;
  \item доходностью различных инвестиционных объектов;
  \item мобильностью трудовых ресурсов;
  \item событиями человеческой жизни (свадьба, развод, рождение ребенка в семье, смена места жительства и др.).
\end{itemize}

Объекты недвижимости занимают значительную часть ресурсов экономики любой страны.
Оценка стоимости – длительный и сложный процесс установления денежного эквивалента стоимости объекта недвижимости.
Она требует высокой квалификации оценщика, владеющего методами и инструментарием оценочной деятельности, знающего состояние
рынка недвижимости и особенно нужного сегмента, детального значения правовых особенностей сделок с недвижимостью и др~\cite{audit}.
Практика показывает, что для оценки стоимости объекта недвижимости, специалисту требуется значительное время.

С развитием теоретических подходов для создания адекватных моделей поведения рынка недвижимости в западных странах и
США одновременно происходило активное внедрение новых интеллектуальных компьютерных технологий в практику принятия
финансовых и инвестиционных решений. Вначале в виде экспертных систем и баз знаний, а затем с конца 80-х - нейросетевых
технологий, которые являются адекватным аппаратом для решения задач прогнозирования.

Начало исследования методов обработки информации, называемых сегодня нейросетевыми, было положено несколько десятилетий
назад. С течением времени интерес к нейросетевым технологиям то ослабевал, то вновь возрождался. Такое непостоянство
напрямую связано с практическими результатами проводимых исследований.

Цена на жилье - вопрос крайне сложный, изучение основных факторы влияния и выяснение правил изменения имеют важное теоретическое и
практическое значение для способствования устойчивому и здоровому развитию рынка жилой недвижимости.
Автоматизация позволит ускорить процесс принятия решения, учесть
большее количество факторов и снизить уровень субъективности.


% Целью исследования является проверка адекватности применения методов эконометрического
% анализа для оценки объектов недвижимости и построение на их основе модели стоимости, проектирование нейронной сети для
% прогнозирования стоимости недвижимости, сравнение полученных результатов с целью выявления более точной модели и, как
% следствие, прогнозирование более точной стоимости. Результаты исследования могут быть полезны для прогнозирования
% ценообразования на рынке недвижимости, а также при оценке стоимости объектов недвижимости.
