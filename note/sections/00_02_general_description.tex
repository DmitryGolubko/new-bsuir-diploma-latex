\sectioncentered*{ОБЩАЯ ХАРАКТЕРИСТИКА РАБОТЫ}
\addcontentsline{toc}{section}{ОБЩАЯ ХАРАКТЕРИСТИКА РАБОТЫ}
\label{sec:general_description}

\textbf{Цель и задачи исследования}\\

Цель диссертационной работы – провести анализ применяемых моделей и алгоритмов, которые используются при
прогнозировании цен на рынке недвижимости, проверить адекватность применения методов эконометрического
анализа для оценки объектов недвижимости и построение на их основе модели стоимости, применить существующие алгоритмы на
подготовленной выборке данных.
На основе проведенного анализа выполнить сравнение алгоритмов, определить наиболее эффективный из них.
Результаты исследования могут быть полезны для прогнозирования
ценообразования на рынке недвижимости, а также при оценке стоимости объектов недвижимости.

Для достижения поставленной цели необходимо решить следующие задачи:
\begin{enumerate}
  \item Провести анализ применяемых моделей и алгоритмов, которые используются при прогнозировании цен на рынке недвижимости.
  \item Реализовать предложенные модели и провести экспериментальные исследования на основе накопленных данных.
  \item На основе сравнительного анализа выбрать более эффективный и достоверный.
\end{enumerate}

\textit{Объектом} исследования выступает процесс формирования цен \linebreak на недвижимость.

\textit{Предметом} исследования являются модели и алгоритмы, которые используются для анализа процесса формирования цен.

Основной \textit{гипотезой}, положенной в основу диссертационной работы, является сравнение результатов работы выбранных алгоритмов
и выявление более эффективного способа оценки стоимости недвижимости.
~\\

\textbf{Структура и объем диссертации}\\

Диссертация состоит из введения, общей характеристики работы, двух глав, заключения, списка использованных источников,
списка публикаций автора и приложений. В первой главе представлен анализ cуществующих применяемых моделей и алгоритмов.
Вторая глава посвящена проведению эксперименального исследования, реализацией выбранных алгоритмов анализа, проведению
их сравнительного анализа.
