\sectioncentered*{ОБЩАЯ ХАРАКТЕРИСТИКА РАБОТЫ}
\addcontentsline{toc}{section}{ОБЩАЯ ХАРАКТЕРИСТИКА РАБОТЫ}
\label{sec:general_description}

\textbf{Цель и задачи исследования}\\

Цель диссертационной работы – провести анализ применяемых моделей и алгоритмов, которые используются при
прогнозировании цен на рынке недвижимости, проверить адекватность применения методов эконометрического
анализа для оценки объектов недвижимости и построение на их основе модели стоимости, применить существующие алгоритмы на
подготовленной выборке данных.
На основе проведенного анализа выполнить сравнение алгоритмов, создать соответствующее ПС для управления объектами
недвижимости с возможностью предварительной оценки стоимости недвижимости.
Результаты исследования могут быть полезны для прогнозирования
ценообразования на рынке недвижимости, а также при оценке стоимости объектов недвижимости.

Для достижения поставленной цели необходимо решить следующие задачи:
\begin{enumerate}
  \item Провести анализ применяемых моделей и алгоритмов, которые используются при прогнозировании цен на рынке недвижимости.
  \item Реализовать предложенные модели и провести экспериментальные исследования на основе накопленных данных.
  \item На основе сравнительного анализа выбрать более эффективный и достоверный.
  \item Реализовать соответствующее ПС для управления и предварительной оценки стоимости недвижимости.
\end{enumerate}
% Объектом исследования выступает программное обеспечение систем прогнозирования распространения радионуклидов в окружающей среде
% Предметом исследования являются модели и алгоритмы, которые используются в системах моделирования для аварийных
% ситуаций, связанных с выбросом загрязняющих веществ в окружающую среду.
% Основной гипотезой, положенной в основу диссертационной работы, является возможность использования программной
% реализации улучшенной локальной модели распространения загрязняющих веществ в окружающей среде для поддержки принятия 
% экспертного решения в первые часы развития аварийной ситуации на ПОО, когда существует дефицит данных активных измерений
% и значений наиболее критических параметров для более сложных моделей.

\textbf{Структура и объем диссертации}\\

Диссертация состоит из введения, общей характеристики работы, трех глав, заключения, списка использованных источников,
списка публикаций автора и приложений. В первой главе представлен анализ cуществующих применяемых моделей и алгоритмов.
Вторая глава посвящена проведению эксперименального исследования, реализацией выбранных алгоритмов анализа, проведению
их сравнительного анализа. 
В третьей главе предложена разработка программной реализации системы управления недвижимости с возможностью предварительной
оценки стоимости объектов недвижимости.
