\subsection{Регрессионный анализ}
\label{sec:analogues:regression}

Исходя из проведенного анализа способов применения различных методов регрессии можно сделать вывод, что наиболее
подходящим способом для анализа процесса формирования цен на недвижимость является множественный линейный регрессионный
анализ.

Данное утверждение подтверждается анализом статьи~\cite{using_regression_analysis}.

В целях оценки недвижимости может применяться либо многофакторный, либо однофакторный регрессионный анализ.
В первом случае строится множественная регрессионная модель, описывающая зависимость стоимости оцениваемого объекта от
нескольких независимых определяющих факторов, значения которых определяются из анализа рыночных данных.
Этими факторами могут быть как физические характеристики объекта (площадь, качество отделки и т.п.), так и
характеристики его местоположения (удаленность от транспортных магистралей, экологическая обстановка и т.п.).

При однофакторном регрессионном анализе рассматривается зависимость переменной – стоимости единицы сравнения – от одной
независимой (контролируемой) переменной. Значения остальных независимых переменных считаются фиксированными.
В качестве независимой переменной X обычно используется показатель «общая площадь», за зависимую переменную Y
принимается показатель «стоимость 1м кв. общей площади».

В случае, когда рассматривается зависимость между одной зависимой переменной Y и несколькими независимыми
переменными Х1, Х2, ... ,Хn говорят о множественной линейной регрессии. 

В данной статье достаточно хорошо показан объем данных и набор парамтерв, используемых в дальнейшем анализе, построена
отдельные модели по количеству комнат в квартире.

Однако в данной статье проведен анализ рынка первичного жилья, который не всегда отражает реальную рыночную стоимость.
Также отсутствует общая модель с учетом всех параметров, по которой невозможно сделать сравнение с отдельными моделями
по количеству комнат.

Данные недостатки решено устранить, помимо отдельных моделей по количеству комнат, принято решение также построить
дополнительно модели по району города, в котором продается объект недвижимости и провести сравнение полученных результатов
с целью выбора наилучшей модели.
