\subsection{Требования к проектируемому программному средству}
\label{sec:analysis:specification}

По результатам изучения предметной области, анализа литературных источников и обзора существующих систем-аналогов сформулируем требования к проектируемому программному средству.

\subsubsection{} Назначение проекта
\label{sec:analysis:specification:purpose}

Назначением проекта является разработка программного средства, автоматизирующего основные задачи участников учебного процесса университетов: управление расписанием, заданиями и коммуникацией.

\subsubsection{} Основные функции
\label{sec:analysis:specification:functions}

Программное средство должно поддерживать следующие основные фун\-к\-ции:

\begin{itemize}
	\item регистрация и аутентификация;
	\item поддержка системы ролей;
	\item отображение расписания занятий;
	\item отображение списка изучаемых дисциплин (для студентов) и преподаваемых дисциплин с типами занятий (для преподавателей);
	\item возможность управления индивидуальными заданиями и материалами по дисциплинам;
	\item отправка результатов выполнения индивидуальных заданий;
	\item оценивание/отклонение результатов выполнения заданий;
	\item управление очередями на защиту индивидуальных заданий;
	\item отправка сообщений другим пользователям системы;
	\item управление списками групп, их отображение со всеми выставленными оценками;
	\item возможность проверки посещения занятий.
\end{itemize}

\subsubsection{} Требования к входным данным
\label{sec:analysis:specification:inputs}

Входные данные для программного средства должны быть представлены в виде вводимого пользователем с помощью клавиатуры текста и выбора доступных опций пользовательского интерфейса.

Должны быть реализованы проверки вводимых данных на корректность с отображением информации об ошибках в случае их некорректности.

\subsubsection{} Требования к выходным данным
\label{sec:analysis:specification:outputs}

Выходные данные программного средства должны быть представлены посредством отображения информации с помощью различных элементов пользовательского интерфейса.

\subsubsection{} Требования к временным характеристикам
\label{sec:analysis:specification:timing}

Производительность программно-аппаратного комплекса должна обеспечивать следующие временные характеристики: время реакции не запрос пользователя не должно превышать 1 секунды при минимальной скорости соединения 1 МБит/с. Допускается невыполнение данного требования в случае, когда невозможность обеспечить заявленную производительность обусловлена объективными внешними причинами.

\subsubsection{} Требования к надежности
\label{sec:analysis:specification:reliability}

Надежное функционирование программы должно быть обеспечено выполнением следующих организационно-технических мероприятий:

\begin{itemize}
	\item организация бесперебойного питания;
	\item выполнение рекомендаций Министерства труда и социальной защиты РБ, изложенных в Постановлении от 23 марта 2011 г. «Об утверждении Норм времени на работы по обслуживанию персональных электронно-вычислитель\-ных машин, организационной техники и офисного оборудования»;
	\item выполнение требований ГОСТ 31078-2002 <<Защита информации. Испытания программных средств на наличие компьютерных вирусов>>;
	\item необходимым уровнем квалификации пользователей.
\end{itemize}

Время восстановления после отказа, вызванного сбоем электропитания технических средств (иными внешними факторами), нефатальным сбоем операционной системы, не должно превышать времени, необходимого на перезагрузку операционной системы и запуск программы, при условии соблюдения условий эксплуатации технических и программных средств. Время восстановления после отказа, вызванного неисправностью технических средств, фатальным сбоем операционной системы, не должно превышать времени, требуемого на устранение неисправностей технических средств и переустановки программных средств.

Отказы программы возможны вследствие некорректных действий пользователя при взаимодействии с операционной системой. Во избежание возникновения отказов программы по указанной выше причине следует обеспечить работу конечного пользователя без предоставления ему административных привилегий.

\subsubsection{} Требования к аппаратному обеспечению серверной части
\label{sec:analysis:specification:server_requirments}

ЭВМ, на которой должна функционировать серверная часть программного средства, должна обладать следующими минимальными характеристиками:

\begin{itemize}
	\item процессор Intel Core i5 с тактовой частотой 2 ГГц;
	\item жесткий диск объемом 100 Гб;
	\item оперативная память 4 Гб;
	\item сетевая карта Ethernet 100 МБит/с.
\end{itemize}

Также для функционирования серверной части требуется установленный Apache HTTP Server, который является кроссплатформенным программные средством, вследствие чего вопрос о целевой операционной системе не рассматривается. Кроме того, процедуры установки и настройки данного веб-сервера выходят за рамки данного проекта и также не рассматриваются.

\subsubsection{} Требования к аппаратному обеспечению клиентской части
\label{sec:analysis:specification:client_requirments}

Клиентская часть программного средства должна функционировать на ЭВМ со следующими минимальными характеристиками:

\begin{itemize}
	\item процессор Intel Pentium 4 с тактовой частотой 2 ГГц и более;
	\item оперативная память 2 Гб и более;
	\item сетевая карта Ethernet 10/100 Мбит.
\end{itemize}

Для корректной работы программного средства необходим один из следующих браузеров с соответствующей минимальной версией:

\begin{itemize}
	\item Google Chrome 49;
	\item Vivaldi 1.0;
	\item Opera 34;
	\item Mozilla Firefox 43;
	\item Apple Safari 9.0;
	\item Microsoft Edge 20.
\end{itemize}

\subsubsection{} Выбор технологий программирования
\label{sec:analysis:specification:language}

Язык программирования, на котором будет реализована система, заслуживает большого внимания, так как вы будете погружены в него с начала конструирования программы до самого конца. Исследования показали, что выбор языка программирования несколькими способами влияет на производительность труда программистов и качество создаваемого ими кода. Если язык хорошо знаком программистам, они работают более производительно. Данные, полученные при помощи модели оценки Cocomo II, показывают, что программисты, использующие язык, с которым они работали три года или более, примерно на 30\% более продуктивны, чем программисты, обладающие аналогичным опытом, но для которых язык является новым~\cite{software_cost_estimation}. В более раннем исследовании, проведенном в IBM, было обнаружено, что программисты, обладающие богатым опытом использования языка программирования, были более чем втрое производительнее программистов, имеющих минимальный опыт~\cite{method_of_programming_measurement_and_estimation}.

Язык программирования \typescript, указанный в задании на дипломное проектирование, является кроссплатформенным языком программирования. Данный ЯП представляет собой надмножество языка JavaScript, что означает, что их объединяют общие синтаксис и семантика управляющих конструкций; ключевой отличительной особенностью является возможность использования строгой типизации, что значительно упрощает статические проверки кода. Кроме того, он компилируется в обычный JavaScript, что означает возможность запуска кода в любом браузере или движке, который поддерживает стандарт ECMAScript 3 или более новый \cite{typescript}. 

Несмотря на то, что \typescript, так же как и JavaScript, изначально предназначался для запуска в браузере клиента, в настоящее время разработаны фреймворки, позволяющие использовать его для разработки под различными платформами: Windows 10 \cite{modern_apps}, iOS, Android \cite{nativescript} и другие.

Исходя из достоинств данного языка программирования, можно сделать вывод, что он наиболее подходящий для решения проблем, схожих с поднимаемыми в данной пояснительной записке. Именно поэтому \typescript и выбран как основной язык программирования в задании к текущему дипломному проекту.

Однако, выбранный язык программирования является средством для программирования клиентской части приложения. Поскольку для приложения в любом случае понадобится база данных, то есть два варианта: 

\begin{itemize}
	\item осуществлять запросы к БД напрямую с клиентского приложения;
	\item реализовать серверную прослойку между клиентской частью и базой данных.
\end{itemize}

Первый подход крайне небезопасен: очень опасно предоставлять открытый доступ к БД. В это же время, второй способ, помимо ее сокрытия, предоставляет возможность проверки подлинности и предоставления прав пользователям. В связи с этим появляется проблема выбора технологий для серверной части. Основным влияющим фактором является имеющийся опыт команды разработки, в связи с чем была выбрана технология .Net и язык программирования \csharp. 

.NET Framework — программная платформа, выпущенная компанией Mi\-c\-ro\-soft в 2002 году. Она была призвана решить ряд наболевших проблем в мире разработки ПО, скопившихся на момент ее выхода, что отражено в целях, которые
были поставлены в ходе ее разработке. 

При разработке платформы .NET учитывались следующие цели~\cite{msdn_dotnet}:

\begin{itemize}
	\item Обеспечение согласованной объектно-ориентированной среды про\-г\-ра\-мми\-ро\-ва\-ния для локального сохранения и выполнения объектного кода, для локального выполнения кода, распределенного в Интернете, либо для удаленного выполнения.
	\item Обеспечение среды выполнения кода, минимизирующей конфликты при развертывании программного обеспечения и управлении версиями.
	\item Обеспечение среды выполнения кода, гарантирующей безопасное выполнение кода, включая код, созданный неизвестным или не полностью доверенным сторонним изготовителем.
	\item Обеспечение среды выполнения кода, исключающей проблемы с производительностью сред выполнения сценариев или интерпретируемого кода.
	\item Обеспечение единых принципов работы разработчиков для разных типов приложений, таких как приложения Windows и веб-приложения.
	\item Разработка взаимодействия на основе промышленных стандартов, которое обеспечит интеграцию кода платформы .NET Framework с любым другим кодом.
\end{itemize}

Несмотря на то, что платформа .Net поддерживает несколько языков программирования, основным является язык \csharp. Он является простым, современным, объектно-ориентированным, обеспечивающим безопасность типов языком программирования. Он соответствует международному стандарту Европейской ассоциации производителей компьютеров — стандарт ECMA-334, а также стандарту Международной организации по стандартизации (In\-ter\-na\-ti\-o\-nal Standards Organization, ISO) и Международной электротехнической комиссии — стандарт ISO/IEC 23270. Компилятор Microsoft \csharp для .NET Framework согласуется с обоими этими стандартами~\cite{msdn_charp}.

Одной из сред программирования, которая поддерживает одновременно и \csharp, и \typescript, является Microsoft Visual Studio, которая входит в линейку продуктов компании Microsoft, включающих интегрированную среду разработки программного обеспечения и ряд других инструментальных средств. Она включает в себя редактор исходного кода с поддержкой технологии In\-tel\-li\-Sen\-se и возможностью рефакторинга кода. Встроенный отладчик может работать как отладчик уровня исходного кода, так и как отладчик машинного уровня. Visual Studio позволяет создавать и подключать сторонние дополнения для расширения функциональности практически на каждом уровне, включая добавление поддержки систем контроля версий исходного кода, добавление новых наборов инструментов или инструментов для прочих аспектов процесса разработки программного обеспечения. Именно поэтому она и выбрана в качестве основной среды программирования.

Язык программирования \typescript~можно использовать для создания приложений для различных платформ. Для проектируемого программного сре\-д\-с\-т\-ва актуальны следующие характеристики:
\begin{itemize}
  \item нет необходимости в организации ресурсоемких вычислений;
  \item желательна возможность использования мгновенных уведомлений и оповещений;
  \item желательна доступность приложения на различных устройствах.
\end{itemize}

По результатам обзора возможных платформ, представленных в пункте~\ref{sec:analysis:literature:platforms}, было принято решение выбрать основной для разработки платформу веб-приложений. После завершения разработки первой версии программного средства будет рассматриваться вопрос разработки мобильного приложения.

Фактор опыта использования оказал влияние на выбор системы управления базами данных для разрабатываемого приложения. СУБД \nezaboodka является особенно приспособленной для веб-приложений. Ее отличительной особенностью является возможность масштабируемости, а также отказоустойчивость. Основным способом взаимодействия с данной СУБД является предложенная разработчиком клиентская библиотека на языке \csharp.

Сформулированные требования позволят осуществить успешное проектирование и разработку программного средства.
