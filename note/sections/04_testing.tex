\section{Тестирование и проверка работоспособности программного средства}
\label{sec:testing}

Тестирование программного обеспечения -- процесс анализа программного средства и сопутствующей документации с целью
выявления дефектов и повышения качества продукта~\cite{kulikov_testing}. Вот уже несколько десятков лет его стабильно
включают в планы разработки как одна из основных работ, причем выполняемая практически на всех этапах проектов.
Важность своевременного выявления дефектов подчеркивается выявленной эмпирически зависимостью между временем допущения
ошибки и стоимостью ее исправления: график данной функции круто возрастает.

Тестирование можно классифицировать по очень большому количеству признаков. Основные виды классификации включают
следующие~\cite{kulikov_testing}:

\begin{enumerate}
	\item по запуску кода на исполнение:
	\begin{enumerate}
		\item статическое тестирование -- без запуска программного средства;
		\item динамическое тестирование -- с запуском;
	\end{enumerate}
	\item по степени автоматизации:
	\begin{enumerate}
		\item ручное тестирование -- тестовые случаи выполняет человек;
    \item автоматизированное тестирование -- тестовые случаи частично или полностью выполняет специальное
    инструментальное средство;
	\end{enumerate}
	\item по принципам работы с приложением:
	\begin{enumerate}
    \item позитивное тестирование -- все действия с приложением выполняются строго в соответствии с требованиями без
    недопустимых действий или некорректных данных;
    \item негативное тестирование -- проверяется способность приложения продолжать работу в критических ситуациях
    недопустимых действий или данных.
	\end{enumerate}
\end{enumerate}

В данном разделе проведем динамическое ручное тестирование. Его целью является подтверждение соответствия работы
программного средства установленным в начале разработки требованиям. Успешное выполнение приведенных в
данном разделе тестовых случаев должно подтвердить работоспособность программного средства в основных сценариях
использования, а также устойчивость к неверным входным данным. В таблице~\ref{table:testing:positive} приведен список
тестовых случаев, относящихся к позитивному тестированию, в таблице~\ref{table:testing:negative} -- к негативному.

% Зачем: свой счетчик для нумерации тестов.
\newcounter{testnumber}
\newcommand\testnumber{\stepcounter{testnumber}\arabic{testnumber}}

% Переключаем команды нумерации для шагов тестов. В конце файла вернем всё как было.
\renewcommand{\labelenumi}{\arabic{enumi})}
\renewcommand{\labelenumii}{\asbuk{enumii})}

\begin{landscape}
	\begin{longtable}{|>{\centering}m{0.19\textwidth}
					  |p{0.8\textwidth}
					  |p{0.34\textwidth}
					  |>{\centering\arraybackslash}m{0.16\textwidth}|} 
	\caption{Тестовые случаи позитивного тестирования}
  \label{table:testing:positive}\\
	\hline
	\begin{minipage}{1\linewidth}
		\centering Модуль (экран)
	\end{minipage} & 
	\begin{minipage}{1\linewidth}
		\centering Описание тестового случая
	\end{minipage} & 
	\begin{minipage}{1\linewidth}
		\centering Ожидаемые результаты
	\end{minipage} & 
	\centering\arraybackslash Тестовый случай пройден? \endfirsthead

	\caption*{Продолжение таблицы \ref{table:testing:positive}}\\\hline
	\centering 1 & \centering 2 & \centering 3 & \centering\arraybackslash 4 \\\hline \endhead

	\hline
	\centering 1 & \centering 2 & \centering 3 & \centering\arraybackslash 4 \\

  \hline
  Аккаунт &
	\begin{minipage}[t]{1\linewidth}
		\testnumber. \textbf{Регистрация}.\newline
 		Предусловие: необходим существующий ящик электронной почты.
 		\begin{enumerate}
 			\item Нажать кнопку <<Регистрация>> на главной странице ПС.
 			\item Ввести корпоративный адрес электронной почты.
 			\item Ввести пароль 12345678.
 			\item Ввести пароль из предыдущего пункта в поле подтверждения пароля.
 			\item Нажать кнопку <<Зарегистрироваться>>.
 			\item Проверить ящик электронной почты, дождаться получения электронного письма.
 			\item Перейти по ссылке из полученного письма.
 		\end{enumerate}
 	\end{minipage} &
  Отображается страница регистрации. На указанный адрес электронной почты приходит письмо со ссылкой. При переходе по ссылке появляется сообщения <<Аккаунт подтвержден>>.
  & Да \\
	\hline
  \end{longtable}
\end{landscape}

% Зачем: возвращаем нумерацию перечислений как надо по стандарту.
\renewcommand{\labelenumi}{\asbuk{enumi})}
\renewcommand{\labelenumii}{\arabic{enumii})}
