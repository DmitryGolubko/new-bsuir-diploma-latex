\section{Тестирование и проверка работоспособности программного средства}
\label{sec:testing}

Тестирование программного обеспечения -- процесс анализа программного средства и сопутствующей документации с целью выявления дефектов и повышения качества продукта~\cite{kulikov_testing}. Вот уже несколько десятков лет его стабильно включают в планы разработки как одна из основных работ, причем выполняемая практически на всех этапах проектов. Важность своевременного выявления дефектов подчеркивается выявленной эмпирически зависимостью между временем допущения ошибки и стоимостью ее исправления: график данной функции круто возрастает.

Тестирование можно классифицировать по очень большому количеству признаков. Основные виды классификации включают следующие~\cite{kulikov_testing}:

\begin{enumerate}
	\item по запуску кода на исполнение:
	\begin{enumerate}
		\item статическое тестирование -- без запуска программного средства;
		\item динамическое тестирование -- с запуском;
	\end{enumerate}
	\item по степени автоматизации:
	\begin{enumerate}
		\item ручное тестирование -- тестовые случаи выполняет человек;
		\item автоматизированное тестирование -- тестовые случаи частично или полностью выполняет специальное инструментальное средство;
	\end{enumerate}
	\item по принципам работы с приложением:
	\begin{enumerate}
		\item позитивное тестирование -- все действия с приложением выполняются строго в соответствии с требованиями без недопустимых действий или некорректных данных;
		\item негативное тестирование -- проверяется способность приложения продолжать работу в критических ситуациях недопустимых действий или данных.
	\end{enumerate}
\end{enumerate}

В данном разделе проведем динамическое ручное тестирование. Его целью является подтверждение соответствия работы программного средства\linebreak установленным в начале разработки требованиям. Успешное выполнение приведенных в данном разделе тестовых случаев должно подтвердить работоспособность программного средства в основных сценариях использования, а также устойчивость к неверным входным данным. В таблице~\ref{table:testing:positive} приведен список тестовых случаев, относящихся к позитивному тестированию, в таблице~\ref{table:testing:negative} -- к негативному.

% Зачем: свой счетчик для нумерации тестов.
\newcounter{testnumber}
\newcommand\testnumber{\stepcounter{testnumber}\arabic{testnumber}}

% Переключаем команды нумерации для шагов тестов. В конце файла вернем всё как было.
\renewcommand{\labelenumi}{\arabic{enumi})}
\renewcommand{\labelenumii}{\asbuk{enumii})}

\begin{landscape}
	\begin{longtable}{|>{\centering}m{0.19\textwidth}
					  |p{0.8\textwidth}
					  |p{0.34\textwidth}
					  |>{\centering\arraybackslash}m{0.16\textwidth}|} 
	\caption{Тестовые случаи позитивного тестирования}
	\label{table:testing:positive}\\

	\hline
	\begin{minipage}{1\linewidth}
		\centering Модуль (экран)
	\end{minipage} & 
	\begin{minipage}{1\linewidth}
		\centering Описание тестового случая
	\end{minipage} & 
	\begin{minipage}{1\linewidth}
		\centering Ожидаемые результаты
	\end{minipage} & 
	\centering\arraybackslash Тестовый случай пройден? \endfirsthead

	\caption*{Продолжение таблицы \ref{table:testing:positive}}\\\hline
	\centering 1 & \centering 2 & \centering 3 & \centering\arraybackslash 4 \\\hline \endhead

	\hline
	\centering 1 & \centering 2 & \centering 3 & \centering\arraybackslash 4 \\

	\hline
	Аккаунт &
	\begin{minipage}[t]{1\linewidth}
		\testnumber. \textbf{Регистрация}.\newline
 		Предусловие: необходим существующий ящик электронной почты.
 		\begin{enumerate}
 			\item Нажать кнопку <<Регистрация>> на главной странице ПС.
 			\item Ввести адрес электронной почты.
 			\item Ввести пароль 12345678.
 			\item Ввести пароль из предыдущего пункта в поле подтверждения пароля.
 			\item Нажать кнопку <<Зарегистрироваться>>.
 			\item Проверить ящик электронной почты, дождаться получения электронного письма.
 			\item Перейти по ссылке из полученного письма.
 		\end{enumerate}
 	\end{minipage} &
	Отображается страница регистрации. На указанный адрес электронной почты приходит письмо со ссылкой. При переходе по ссылке появляется сообщения <<Аккаунт подтвержден>>. & Да \\
	\hline

	Аккаунт &
	\begin{minipage}[t]{1\linewidth}
		\testnumber. \textbf{Аутентификация}.\newline
		Предусловие: необходим зарегестрированный в системе аккаунт.
		\begin{enumerate}
			\item Нажать кнопку <<Вход>> на главной странице ПС.
			\item Ввести адрес электронной почты и пароль аккаунта.
			\item Нажать кнопку <<Войти>>.
		\end{enumerate}
 	\end{minipage} &
	Отображается страница аутентификации. По нажатию кнопки <<Войти>> открывается главная страница зарегистрированного пользователя. & Да \\

	Аккаунт &
	\begin{minipage}[t]{1\linewidth}
		\testnumber. \textbf{Редактирование профиля}.\newline
		Предусловие: необходимо произвести аутентификацию.
		\begin{enumerate}
			\item Нажать кнопку <<Мой профиль>>.
			\item Нажать кнопку <<Редактировать>>.
			\item Изменить значения полей имени, фамилии, отчества, описания.
			\item Обновить страницу.
			\item Нажать кнопку <<Мой профиль>>.
		\end{enumerate}
 	\end{minipage} &
	Открывается страница профиля пользователя. Открывается страница редактирования личной информации. В профиле отображается новая информация. & Да \\
	\hline

	Роли &
	\begin{minipage}[t]{1\linewidth}
		\testnumber. \textbf{Подача заявки}.\newline
		Предусловие: необходимо наличие зарегистрированного пользователя с ролью администратора факультета.
		\begin{enumerate}
			\item Произвести регистрацию в системе.
			\item Нажать кнопку <<Подача заявки на роль студента>>.
			\item Ввести номер группы, нажать кнопку <<Отправить заявку>>.
			\item Произвести аутентификацию в качестве пользователя с ролью администратора факультета.
			\item Нажать кнопку <<Заявки>>.
			\item Нажать кнопку <<Подтвердить>>.
		\end{enumerate}
 	\end{minipage} &
	Открывается главная страница приложения. Открывается окно подачи заявки. Отображается уведомление об отправке заявки. Заявка отображается у администратора факультета. После подтверждения заявки появляется соответствующее уведомление. & Да \\

	Расписание &
	\begin{minipage}[t]{1\linewidth}
		\testnumber. \textbf{Отображение расписание преподавателя}.\newline
		Примечание: подтверждение роли администратором факультета не рассматривается.
		\begin{enumerate}
			\item Произвести аутентификацию в системе.
			\item Заполнить личную информацию пользователя в соответствии с некоторым существующим преподавателем.
			\item Нажать кнопку <<Подача заявки на роль преподавателя>>.
			\item Выбрать кафедру, нажать кнопку <<Отправить заявку>>.
			\item Дождаться подтверждения заявки.
			\item Нажать кнопку <<Расписание>>.
		\end{enumerate}
 	\end{minipage} &
	Открывается главная страница приложения. Открывается окно подачи заявки. Отображается уведомление об отправке заявки. Отображается расписание преподавателя. & Да \\
	\hline

	Сообщения &
	\begin{minipage}[t]{1\linewidth}
		\testnumber. \textbf{Передача сообщений}.\newline
		Предусловие: необходимо наличие двух зарегистрированных пользователей.
		\begin{enumerate}
			\item Произвести аутентификацию в системе.
			\item Нажать кнопку <<Сообщения>>, затем <<Новое сообщение>>.
			\item С помощью поля поиска найти второго пользователя, которому будет отправлено сообщение.
			\item Ввести некоторый текст ненулевой длины, нажать кнопку <<Отправить>>.
			\item Произвести аутентификацию в качестве другого пользователя.
			\item Нажать кнопку <<Сообщения>>.
			\item Выбрать диалог с первым пользователем.
		\end{enumerate}
 	\end{minipage} &
	Открывается экран диалогов. Открывается окно выбора пользователя для отправки сообщения. Есть возможность поиска и выбора пользователей. Созданное сообщение отображается отправленным.  & Да \\

	Прогресс студента &
	\begin{minipage}[t]{1\linewidth}
		\testnumber. \textbf{Отображение прогресса студента}.\newline
		Предусловие: наличие зарегистрированного пользователя с ролью студента, созданные для его группы индивидуальные задания. 
		\begin{enumerate}
			\item Произвести аутентификацию в системе.
			\item Нажать кнопку <<Расписание>>.
			\item Выбрать строку с нелекционным занятием.
			\item Выбрать индивидуальное задание.
			\item Нажать кнопку <<Предметы>>.
			\item Выбрать строку с некоторым предметом.
			\item Выбрать индивидуальное задание.
		\end{enumerate}
 	\end{minipage} &
	Отображается расписание студента. В строках практических и лабораторных занятий с помощью специальных символов и цветов отображается прогресс выполнения заданий. В правой части отображается экран с деталями данного задания. Отображается список дисциплин студента. В его строках отображается прогресс выполнения заданий. В правой части отображается экран с деталями выбранного задания. & Да \\

	Материалы &
	\begin{minipage}[t]{1\linewidth}
		\testnumber. \textbf{Редактирование материалов преподавателя}.\newline
		Предусловие: наличие зарегистрированного пользователя с ролью преподавателя. 
		\begin{enumerate}
			\item Произвести аутентификацию в системе.
			\item Нажать кнопку <<Предметы>>.
			\item Выбрать предмет из списка.
			\item Нажать кнопку <<Добавить>> в области материалов.
			\item Выбрать файл материалов для загрузки.
			\item Повторить добавление материалов не менее трех раз.
			\item Нажать кнопку <<Удалить>> у одного из загруженных материалов.
		\end{enumerate}
 	\end{minipage} &
	Отображается список предметов преподавателя. В правой части отображается экран с детальной информацией о предмете. Открывается стандартный диалог загрузки файлов. Загруженные файлы немедленно отображаются в списке материалов. Удаленный материал немедленно исчезает. & Да \\
	\hline

	Задания &
	\begin{minipage}[t]{1\linewidth}
		\testnumber. \textbf{Создание индивидуальных заданий}.\newline
		Предусловие: наличие зарегистрированного пользователя с ролью преподавателя. 
		\begin{enumerate}
			\item Произвести аутентификацию в системе.
			\item Нажать кнопку <<Предметы>>.
			\item Выбрать предмет из списка.
			\item Нажать кнопку <<Добавить>> в области индивидуальных заданий.
			\item Выбрать файл условия для загрузки.
		\end{enumerate}
 	\end{minipage} &
	Отображается список предметов преподавателя. В правой части отображается экран с детальной информацией о предмете. Открывается стандартный диалог загрузки файлов. Задание немедленно отображается в списке. & Да \\

	Результаты &
	\begin{minipage}[t]{1\linewidth}
		\testnumber. \textbf{Отправка результатов выполнения заданий}.\newline
		Предусловие: наличие зарегистрированного пользователя с ролью студента и созданного для него задания. 
		\begin{enumerate}
			\item Произвести аутентификацию в системе.
			\item Нажать кнопку <<Предметы>>.
			\item Выбрать предмет из списка.
			\item В области "Результаты" нажать кнопку <<Добавить>>.
			\item Выбрать файл для загрузки.
			\item Нажать кнопку <<Отправить>>.
			\item Дождаться подтверждения выполнения задания.
		\end{enumerate}
 	\end{minipage} &
	Отображается список предметов студента. В правой части отображается экран с детальной информацией о предмете. Открывается стандартный диалог загрузки файлов. После отправки и проверки преподавателем появляется оценка. & Да \\
	\hline

	Результаты &
	\begin{minipage}[t]{1\linewidth}
		\testnumber. \textbf{Проверка результатов выполнения заданий}.\newline
		Предусловие: наличие зарегистрированных пользователей с ролями преподавателя и студента. Студент отправил результаты не проверку.
		\begin{enumerate}
			\item Произвести аутентификацию в системе.
			\item Нажать кнопку <<Очередь проверки>>.
			\item Выбрать полученные результаты.
			\item Выбрать и скачать полученный файл с результатами.
			\item Выставить оценку.
		\end{enumerate}
 	\end{minipage} &
	Отображается список предметов преподавателя. В правой части отображается экран с детальной информацией о предмете. Скачивается файл с результатами выполнения заданий. Результаты помечаются как проверенные. & Да \\
	\hline

	\end{longtable}


	% Зачем: зануляем счетчик для следующей таблицы.
	\setcounter{testnumber}{0}
	
	\begin{longtable}{|>{\centering}m{0.19\textwidth}
					  |p{0.8\textwidth}
					  |p{0.34\textwidth}
					  |>{\centering\arraybackslash}m{0.16\textwidth}|} 
	\caption{Тестовые случаи негативного тестирования}
	\label{table:testing:negative}\\

	\hline
	\centering Модуль (экран) & \centering Описание тестового случая & \centeringОжидаемые результаты & \centering\arraybackslash Тестовый случай пройден? \endfirsthead

	\caption*{Продолжение таблицы \ref{table:testing:negative}}\\\hline
	\centering 1 & \centering 2 & \centering 3 & \centering\arraybackslash 4 \\\hline \endhead

	\hline
	\centering 1 & \centering 2 & \centering 3 & \centering\arraybackslash 4 \\
	\hline

	Аккаунт &
	\begin{minipage}[t]{1\linewidth}
		\testnumber. \textbf{Повторная регистрация одного email}.\newline
		Предусловие: необходим существующий ящик электронной почты.
		\begin{enumerate}
			\item Произвести регистрацию в системе.
			\item Подтвердить email с помощью ссылки, полученной в электронном письме, отправленном на указанный адрес.
			\item Выйти из системы.
			\item Произвести регистрацию с тем же email.
		\end{enumerate}
 	\end{minipage} &
	Регистрация производится успешно, письмо приходит, при переходе по ссылке отображается сообщение об успешности подтверждения. При попытке регистрации с тем же email появляется сообщение о невозможности регистрации. & Да \\
	\hline

	Аккаунт &
	\begin{minipage}[t]{1\linewidth}
		\testnumber. \textbf{Аутентификации с неправильными данными}.\newline
		Предусловие: необходимо наличие зарегистрированного пользователя.
		\begin{enumerate}
			\item Нажать кнопку <<Вход>> на главной странице приложения.
			\item Ввести email, который заведомо не зарегистрирован в системе и некоторый пароль.
			\item В поле email ввести адрес электронной почты зарегистрированного пользователя.
			\item В поле пароля ввести заведомо неверный пароль.
		\end{enumerate}
 	\end{minipage} &
	Во всех случаях отображается одинаковое сообщение о неверном email или пароле. & Да \\

	% Задания &
	% \begin{minipage}[t]{1\linewidth}
	% 	\testnumber. \textbf{Задание без условия}.\newline
	% 	Предусловие: наличие зарегистрированного пользователя с ролью преподавателя. 
	% 	\begin{enumerate}
	% 		\item Аутентифицироваться в системе.
	% 		\item Нажать кнопку <<Предметы>>.
	% 		\item Выбрать дисциплину из списка.
	% 		\item Нажать кнопку <<Добавить>> в области индивидуальных заданий.
	% 		\item Нажать кнопку <<Готово>>.
	% 	\end{enumerate}
 % 	\end{minipage} &
	%  & Да \\
	% \hline

	Расписание &
	\begin{minipage}[t]{1\linewidth}
		\testnumber. \textbf{Неверный диапазон дат для отображения расписания}.\newline
		Предусловие: необходим зарегистрированный аккаунт пользователя с ролью студента.
		\begin{enumerate}
			\item Аутентифицироваться в системе.
			\item Нажать кнопку <<Расписание>>.
			\item Нажать кнопку выбора начальной даты.
			\item Выбрать дату 15.05.2017.
			\item Нажать кнопку выбора конечной даты.
			\item Выбрать дату 10.05.2017.
		\end{enumerate}
 	\end{minipage} &
	Отображается главная страница приложения. Отображается страница расписания занятий. Появляются выпадающие списки выбора границ диапазона дат. Появляется сообщение о неправильно выбранном интервале дат. & Да \\
	\hline

	Роли &
	\begin{minipage}[t]{1\linewidth}
		\testnumber. \textbf{Неверная группа в заявке студента}.\newline
		Предусловие: необходим зарегистрированный аккаунт пользователя.
		\begin{enumerate}
			\item Аутентифицироваться в системе.
			\item Нажать кнопку <<Подача заявки на роль студента>>.
			\item В качестве группы указать группу 999999.
			\item Нажать кнопку <<Отправить заявку>>.
			\item Дожидаться изменения статуса заявки.
		\end{enumerate}
 	\end{minipage} &
	Отображается главный экран приложения. Появляется окно создания заявки на роль студента. После просмотра заявки администратором факультета статус заявки будет изменен на "Отклонена", поскольку такой группы не существует, что противоречит бизнес-правилам. & Да \\

	Аккаунт &
	\begin{minipage}[t]{1\linewidth}
		\testnumber. \textbf{Заполнение личных данных неверной информацией}.\newline
		Предусловие: необходим существующий адрес электронной почты.
		\begin{enumerate}
			\item Произвести регистрацию в системе с произвольным паролем и существующим адресом электронной почты.
			\item Подтвердить адрес электронной почты путем перехода по ссылке из письма.
			\item Аутентифицироваться в системе.
			\item Нажать кнопку <<Мой профиль>>.
			\item Нажать кнопку <<Редактировать>>.
			\item Проверить, чтобы поля имени и фамилии были пустыми.
			\item Нажать кнопку <<Сохранить>>.
			\item Ввести в поля имени и фамилии строк, содержащих знаки препинания и цифры.
			\item Нажать кнопку <<Сохранить>>.
		\end{enumerate}
 	\end{minipage} &
	Регистрация проходит успешно. На указанный адрес электронной почты приходит письмо с подтверждающей ссылкой. Аутентификация происходит успешно. Открывается страница профиля. Открывается страница редактирования профиля. Появляется сообщение "Имя и фамилия не могут быть пустыми". Появляется сообщение "Введены некорректные имя и/или фамилия". & Да \\
	\hline

	Аккаунт &
	\begin{minipage}[t]{1\linewidth}
		\testnumber. \textbf{Аутентификация без подтверждения}.\newline
		Предусловие: необходим существующий адрес электронной почты.
		\begin{enumerate}
			\item Произвести регистрацию в системе с произвольным паролем и существующим адресом электронной почты.
			\item Нажать кнопку <<Вход>>.
			\item Ввести указанные при регистрации email и пароль.
		\end{enumerate}
 	\end{minipage} &
	Регистрация происходит успешно. При аутентификации появляется сообщение "Данный аккаунт не подтвержден". & Да \\

	\hline
	\end{longtable}
\end{landscape}

% Зачем: возвращаем нумерацию перечислений как надо по стандарту.
\renewcommand{\labelenumi}{\asbuk{enumi})}
\renewcommand{\labelenumii}{\arabic{enumii})}
