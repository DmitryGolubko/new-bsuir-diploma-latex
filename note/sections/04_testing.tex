\section{Тестирование и проверка работоспособности программного средства}
\label{sec:testing}

Тестирование программного обеспечения -- процесс анализа программного средства и сопутствующей документации с целью
выявления дефектов и повышения качества продукта~\cite{kulikov_testing}. Вот уже несколько десятков лет его стабильно
включают в планы разработки как одна из основных работ, причем выполняемая практически на всех этапах проектов.
Важность своевременного выявления дефектов подчеркивается выявленной эмпирически зависимостью между временем допущения
ошибки и стоимостью ее исправления: график данной функции круто возрастает.

Тестирование можно классифицировать по очень большому количеству признаков. Основные виды классификации включают
следующие~\cite{kulikov_testing}:

\begin{enumerate}
  \item по запуску кода на исполнение:
  \begin{enumerate}
    \item статическое тестирование -- без запуска программного средства;
    \item динамическое тестирование -- с запуском;
  \end{enumerate}
  \item по степени автоматизации:
  \begin{enumerate}
    \item ручное тестирование -- тестовые случаи выполняет человек;
    \item автоматизированное тестирование -- тестовые случаи частично или полностью выполняет специальное
    инструментальное средство;
  \end{enumerate}
  \item по принципам работы с приложением:
  \begin{enumerate}
    \item позитивное тестирование -- все действия с приложением выполняются строго в соответствии с требованиями без
    недопустимых действий или некорректных данных;
    \item негативное тестирование -- проверяется способность приложения продолжать работу в критических ситуациях
    недопустимых действий или данных.
  \end{enumerate}
\end{enumerate}

В данном разделе проведем динамическое ручное тестирование. Его целью является подтверждение соответствия работы
программного средства установленным в начале разработки требованиям. Успешное выполнение приведенных в
данном разделе тестовых случаев должно подтвердить работоспособность программного средства в основных сценариях
использования, а также устойчивость к неверным входным данным. В таблице~\ref{table:testing:positive} приведен список
тестовых случаев, относящихся к позитивному тестированию, в таблице~\ref{table:testing:negative} -- к негативному.

% Зачем: свой счетчик для нумерации тестов.
\newcounter{testnumber}
\newcommand\testnumber{\stepcounter{testnumber}\arabic{testnumber}}

% Переключаем команды нумерации для шагов тестов. В конце файла вернем всё как было.
\renewcommand{\labelenumi}{\arabic{enumi})}
\renewcommand{\labelenumii}{\asbuk{enumii})}

\begin{landscape}
  \begin{longtable}{|>{\centering}m{0.19\textwidth}
            |p{0.8\textwidth}
            |p{0.34\textwidth}
            |>{\centering\arraybackslash}m{0.16\textwidth}|} 
    \caption{Тестовые случаи позитивного тестирования}
    \label{table:testing:positive}\\
    \hline
    \begin{minipage}{1\linewidth}
      \centering Модуль (экран)
    \end{minipage} & 
    \begin{minipage}{1\linewidth}
      \centering Описание тестового случая
    \end{minipage} & 
    \begin{minipage}{1\linewidth}
      \centering Ожидаемые результаты
    \end{minipage} & 
    \centering\arraybackslash Тестовый случай пройден? \endfirsthead

    \caption*{Продолжение таблицы \ref{table:testing:positive}}\\\hline
    \centering 1 & \centering 2 & \centering 3 & \centering\arraybackslash 4 \\\hline \endhead

    \hline
    \centering 1 & \centering 2 & \centering 3 & \centering\arraybackslash 4 \\

    \hline
    Аккаунт &
    \begin{minipage}[t]{1\linewidth}
      \testnumber. \textbf{Регистрация}.\newline
      Предусловие: необходим существующий ящик электронной почты.
      \begin{enumerate}
        \item Нажать кнопку <<Регистрация>> на главной странице ПС.
        \item Ввести корпоративный адрес электронной почты.
        \item Ввести пароль 12345678.
        \item Ввести пароль из предыдущего пункта в поле подтверждения пароля.
        \item Нажать кнопку <<Зарегистрироваться>>.
        \item Проверить ящик электронной почты, дождаться получения электронного письма.
        \item Перейти по ссылке из полученного письма.
      \end{enumerate}
    \end{minipage} &
    Отображается страница регистрации. На указанный адрес электронной почты приходит письмо со ссылкой. При переходе по ссылке появляется сообщения <<Аккаунт подтвержден>>. & Да \\
    \hline

    Аккаунт &
    \begin{minipage}[t]{1\linewidth}
      \testnumber. \textbf{Аутентификация}.\newline
      Предусловие: необходим зарегестрированный в системе аккаунт.
      \begin{enumerate}
        \item Ввести адрес электронной почты и пароль аккаунта.
        \item Нажать кнопку <<Войти>>.
      \end{enumerate}
    \end{minipage} &
    Отображается страница аутентификации. По нажатию кнопки <<Войти>> открывается страница заполнения профиля пользователя. & Да \\

    Профиль &
    \begin{minipage}[t]{1\linewidth}
      \testnumber. \textbf{Заполнение профиля}.\newline
      Предусловие: необходим аутентифицированный в системе аккаунт.
      \begin{enumerate}
        \item Ввести данные для заполнения профиля.
        \item Нажать кнопку <<отправить на проверку>>.
      \end{enumerate}
    \end{minipage} &
    Отображается страница профиля. Профиль отправляется для проверки техническому менеджеру. Появляется всплывающее сообщение. & Да \\
    \hline

    Аккаунт &
    \begin{minipage}[t]{1\linewidth}
      \testnumber. \textbf{Проверка профиля менеджером}.\newline
      Предусловие: необходим аутентифицированный в системе аккаунт, профиль, требующий проверки, соответствующие права пользователя.
      \begin{enumerate}
        \item Открыть профиль, требующий проверки.
        \item Проверить правильность заполнения профиля.
        \item Если необходимо, внести правки.
        \item Нажать кнопку <<отправить на проверку>>.
      \end{enumerate}
    \end{minipage} &
    Отображается страница профиля. Профиль отправляется для проверки следующему менеджеру. Появляется всплывающее сообщение. Меняется статус профиля на тот, какая проверка требуется. & Да \\
    \hline

    Менеджеры &
    \begin{minipage}[t]{1\linewidth}
      \testnumber. \textbf{Назначение менеджеров}.\newline
      Предусловие: необходим аутентифицированный в системе аккаунт, права администратора.
      \begin{enumerate}
        \item Выбрать из выпадающего списка менеджеров для соответствующих департаментов.
        \item Нажать кнопку <<Сохранить>>.
      \end{enumerate}
    \end{minipage} &
    Отображается страница назначения менеджеров. Выбранные пользователи назначаются менеджерами департаментов. Появляется всплывающее сообщение. & Да \\

    Департаменты &
    \begin{minipage}[t]{1\linewidth}
      \testnumber. \textbf{Создание департамента}.\newline
      Предусловие: необходим аутентифицированный в системе аккаунт, соответствующие права пользователя.
      \begin{enumerate}
        \item Нажать на кнопку <<Создать департамент>>.
        \item Ввести название департамента.
        \item Выбрать тип офиса для нового департамента из выпадающего списка.
        \item Нажать кнопку <<Создать новый департамент>>.
      \end{enumerate}
    \end{minipage} &
    Отображается страница департаментов. В списке появится созданный департамент. Появляется всплывающее сообщение. & Да \\
    \hline

    Проекты &
    \begin{minipage}[t]{1\linewidth}
      \testnumber. \textbf{Создание проекта}.\newline
      Предусловие: необходим аутентифицированный в системе аккаунт, соответствующие права пользователя.
      \begin{enumerate}
        \item Нажать на кнопку <<Создать проект>>.
        \item Ввести название департамента.
        \item Ввести описание проекта.
        \item Выбрать статус проекта.
        \item Выбрать технологии, используемые на проекте.
        \item Нажать кнопку <<Добавить пользователя>>.
        \item Выбрать пользователя из списка.
        \item Ввести его роль и зону ответственности
        \item Нажать кнопку <<Создать новый проект>>.
      \end{enumerate}
    \end{minipage} &
    Отображается страница созданного проекта. Появляется всплывающее сообщение. & Да \\

    Позиции &
    \begin{minipage}[t]{1\linewidth}
      \testnumber. \textbf{Создание позиции}.\newline
      Предусловие: необходим аутентифицированный в системе аккаунт, соответствующие права пользователя.
      \begin{enumerate}
        \item Нажать на кнопку <<Создать позицию>>.
        \item Ввести название позиции.
        \item Выбрать департамент для позиции.
        \item Выбрать технологии для позиции.
        \item Нажать кнопку <<Создать новую позицию>>.
      \end{enumerate}
    \end{minipage} &
    Отображается страница позиции. Появляется всплывающее сообщение. & Да \\
    \hline

    Вопросы &
    \begin{minipage}[t]{1\linewidth}
      \testnumber. \textbf{Создание вопроса}.\newline
      Предусловие: необходим аутентифицированный в системе аккаунт, соответствующие права пользователя.
      \begin{enumerate}
        \item Нажать на кнопку <<Создать вопрос>>.
        \item Ввести вопрос.
        \item Ввести краткое описание для резюме.
        \item Ввести пример ответа.
        \item Выбрать позиции, для которых будет доступен создаваемый вопрос.
        \item Нажать кнопку <<Создать новый вопрос>>.
      \end{enumerate}
    \end{minipage} &
    Отображается страница вопроса. Появляется всплывающее сообщение. & Да \\
    \hline

    Профиль &
    \begin{minipage}[t]{1\linewidth}
      \testnumber. \textbf{Генерация резюме}.\newline
      Предусловие: необходим аутентифицированный в системе аккаунт, соответствующие права пользователя.
      \begin{enumerate}
        \item Открыть профиль пользователя.
        \item Нажать на кнопку <<Генерировать резюме>>..
      \end{enumerate}
    \end{minipage} &
    Отображается сгенерированная страница резюме. & Да \\
    \hline
  \end{longtable}

  % Зачем: зануляем счетчик для следующей таблицы.
  \setcounter{testnumber}{0}

  \begin{longtable}{|>{\centering}m{0.19\textwidth}
            |p{0.8\textwidth}
            |p{0.34\textwidth}
            |>{\centering\arraybackslash}m{0.16\textwidth}|} 
    \caption{Тестовые случаи негативного тестирования}
    \label{table:testing:negative}\\

    \hline
    \centering Модуль (экран) & \centering Описание тестового случая & \centeringОжидаемые результаты & \centering\arraybackslash Тестовый случай пройден? \endfirsthead

    \caption*{Продолжение таблицы \ref{table:testing:negative}}\\\hline
    \centering 1 & \centering 2 & \centering 3 & \centering\arraybackslash 4 \\\hline \endhead

    \hline
    \centering 1 & \centering 2 & \centering 3 & \centering\arraybackslash 4 \\
    \hline

    Аккаунт &
    \begin{minipage}[t]{1\linewidth}
      \testnumber. \textbf{Повторная регистрация одного email}.\newline
      Предусловие: необходим существующий ящик электронной почты.
      \begin{enumerate}
        \item Произвести регистрацию в системе.
        \item Подтвердить email с помощью ссылки, полученной в электронном письме, отправленном на указанный адрес.
        \item Выйти из системы.
        \item Произвести регистрацию с тем же email.
      \end{enumerate}
    \end{minipage} &
    Регистрация производится успешно, письмо приходит, при переходе по ссылке отображается сообщение об успешности подтверждения. При попытке регистрации с тем же email появляется сообщение о невозможности регистрации. & Да \\
    \hline

    Аккаунт &
    \begin{minipage}[t]{1\linewidth}
      \testnumber. \textbf{Аутентификации с неправильными данными}.\newline
      Предусловие: необходимо наличие зарегистрированного пользователя.
      \begin{enumerate}
        \item Открыть главную страницу приложения.
        \item В поле email ввести адрес электронной почты зарегистрированного пользователя.
        \item В поле пароля ввести заведомо неверный пароль.
      \end{enumerate}
     \end{minipage} &
    Во всех случаях отображается одинаковое сообщение о неверном email или пароле. & Да \\

    Аккаунт &
    \begin{minipage}[t]{1\linewidth}
      \testnumber. \textbf{Аутентификация без подтверждения}.\newline
      Предусловие: необходим существующий адрес электронной почты.
      \begin{enumerate}
        \item Произвести регистрацию в системе с произвольным паролем и существующим адресом электронной почты.
        \item Открыть страницу входа
        \item Ввести указанные при регистрации email и пароль.
        \item Нажать кнопку <<Вход>>.
      \end{enumerate}
     \end{minipage} &
    Регистрация происходит успешно. При аутентификации появляется сообщение "Данный аккаунт не подтвержден". & Да \\
    \hline

    Менеджеры &
    \begin{minipage}[t]{1\linewidth}
      \testnumber. \textbf{Назначение не всех менеджеров}.\newline
      Предусловие: необходимо наличие зарегистрированного пользователя, обладающего соответствующими правами.
      \begin{enumerate}
        \item Открыть страницу входа
        \item Ввести указанные при регистрации email и пароль.
        \item Нажать кнопку <<Вход>>.
        \item Открыть страницу назначения менеджеров.
        \item Выбрать не всех менеджеров для департаментов.
        \item Нажать кнопку <<Сохранить>>.
      \end{enumerate}
     \end{minipage} &
     Аутентификация происходит успешно. При назначении менеджеров повляется всплывающее сообщение об ошибке, изменения не сохраняются. & Да \\

    Профиль &
      \begin{minipage}[t]{1\linewidth}
        \testnumber. \textbf{Заполнение личных данных неверной информацией}.\newline
        Предусловие: необходим существующий адрес электронной почты.
        \begin{enumerate}
          \item Произвести регистрацию в системе с произвольным паролем и существующим адресом электронной почты.
          \item Подтвердить адрес электронной почты путем перехода по ссылке из письма.
          \item Аутентифицироваться в системе.
          \item Нажать кнопку <<Редактировать профиль>>.
          \item Проверить, чтобы поля имени и фамилии были пустыми.
          \item Оставить секцию ответа на вопросы пустой. Не заполнить данные об уровне английского.
          \item Нажать кнопку <<Сохранить>>.
        \end{enumerate}
      \end{minipage} &
      Регистрация проходит успешно. На указанный адрес электронной почты приходит письмо с подтверждающей ссылкой. Аутентификация происходит успешно. Открывается страница профиля. Открывается страница редактирования профиля. Появляется сообщение "Имя и фамилия не могут быть пустыми". Подсветят поля ответа на вопросы, уровень английского языка. & Да \\
      \hline
  \end{longtable}
\end{landscape}

% Зачем: возвращаем нумерацию перечислений как надо по стандарту.
\renewcommand{\labelenumi}{\asbuk{enumi})}
\renewcommand{\labelenumii}{\arabic{enumii})}
