\subsection{Сравнение результатов}
\label{sec:experiment:conclusion}

На основании проведенных исследований можно утверждать, что применение нейронных сетей для прогнозирования
стоимости объектов недвижимости более эффективно использования методов регрессионного анализа и может достаточно
точно отражать рыночную стоимость недвижимости. Однако с увеличением числа выборки точность прогнозирования падает ввиду
присутствия множества скрытых факторов(такие как время постройки дома, удаленность от различных сервисов, наличие или
отсутствие мебели, качество постройки дома), не учитываемых в данном исследовании.
Предложенные методы могут быть использованы продавцами для первичной оценки стоимости жилой недвижимости,
а покупателями могут использоваться в качестве дополнительного источника информации, однако данная модель требует
доработки.
