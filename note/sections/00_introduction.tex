\sectioncentered*{Определения и сокращения}
\label{sec:definitions}

В настоящей пояснительной записке применяются следующие определения и сокращения.
\\

\emph{Спецификация} -- документ, который желательно полно, точно и верифицируемо определяет требования, дизайн,
поведение или другие характеристики компонента или системы, и, часто, инструкции для контроля выполнения этих
требований \cite{istqb_specification}.

\emph{Веб-приложение} -- клиент-серверное приложение, в котором клиентом выступает браузер, а сервером – веб-сервер.

\emph{Кроссплатформенность} -- способность программного обеспечения работать более чем на одной аппаратной платформе и (или) операционной системе.

\emph{CSS} -- Cascading Style Sheets -- каскадные таблицы стилей -- формальный язык описания внешнего вида документа,
написанного с использованием языка разметки

\emph{HTML} -- HyperText Markup Language -- язык гипертекстовой разметки -- стандартизированный язык разметки документов
во Всемирной паутине.

\emph{API} -- Application Programming Interface -- программный интерфейс приложения -- описание способов, которыми одна компьютерная программа может
взаимодействовать с другой программой.

\emph{JSON} -- текстовый формат обмена данными, основанный на JavaScript.

ПС -- программное средство

ПО -- программное обеспечение.

БД -- база данных.

СУБД -- система управления базами данных.

HR-менеджмент -- управление персоналом.
\\

\sectioncentered*{Введение}
\addcontentsline{toc}{section}{ВВЕДЕНИЕ}
\label{sec:introduction}
 
Развитие информационных технологий повлияло и на такую сферу деятельности компании, как управление персоналом.
Нельзя не отметить, что управление персоналом является одним из наиболее важных составляющих компании, так как
грамотное управление персоналом способно повысить эффективность деятельности сотрудников компании и увеличить
прибыль. Впоследнее время количество компаний, которые хотят автоматизировать процесс управления персоналом при помощи
информационных систем, значительно увеличилось. Такие информационные системы помогают достичь целей компании в более
короткие сроки и без дополнительных финансовых вложений.

На сегодняшний день использование информационных технологий вуправлении персоналом - это необходимое условие для
того, чтобы обеспечить эффективную работу любой компании. Как показывает практика, одному менеджеру кадровой службы
при помощи автоматизированных систем под силу вести дела сотни сотрудников компании.

Программы, которые автоматизируют определенные участки деятельности кадровой службы, дают возможность проводить
отбор, аттестацию и учет работников; разрабатывать штатное расписание; рассчитывать заработную плату; составлять
аналитические отчеты тенденций предприятия. Данный вид программ используется в небольших по размерам компаниях
для того, чтобы решать отдельные задачи \cite{cyberleninka}.

В данном дипломном проекте реализуется программное средство менеджмента персонала, которое способно отслеживать
активные вакансии в компании, текущую занятость сотрудников на проектах, их способности, а также на основе полученных
данных автоматически генерировть резюме сотрудника для дальнейшего использования в отделе маркетинга и продаж.

Целью дипломного проекта является разработка и реализация кроссплатформенного веб-приложения, сообщающегося с сервером
в сети Интернет с целью предоставления пользователям верной и актуальной информации о сотрудниках компании. В первую
очередь разрабатываемое программное средство предназначается для работников отдела маркетинга, поскольку они наиболее
часто запрашивают актуальную информацию о сотрудниках в удобном для представления формате с целью передачи данных заказчикам.

В пояснительной записке к дипломному проекту излагаются детали поэтапной разработки приложения менеджмента персонала.
В первом разделе приведены результаты анализа литературных источников по теме дипломного проекта, рассмотрены
особенности существующих систем-аналогов, выдвинуты требования к проектируемому ПС. Во втором разделе приведено
описание функциональности проектируемого ПС, представлена спецификация функциональных требований. В третьем
разделе приведены детали проектирования и конструирования ПС. Результатом этапа конструирования является функционирующее
программное средство. В четвертом разделе представлены доказательства того, что спроектированное ПС работает в
соответствии с выдвинутыми требованиями спецификации. В пятом разделе приведены сведения по развертыванию и запуску ПС,
указаны требуемые аппаратные и программные средства. Обоснование целесообразность создания программного средства с
технико-экономической точки зрения приведено в шестом разделе. Итоги проектирования, конструирования программного
средства, а также соответствующие выводы приведены в заключении.
