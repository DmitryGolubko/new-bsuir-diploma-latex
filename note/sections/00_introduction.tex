\sectioncentered*{Введение}
\addcontentsline{toc}{section}{ВВЕДЕНИЕ}
\label{sec:introduction}

Объекты недвижимости занимают значительную часть ресурсов экономики любой страны. Практика показывает, что для оценки
стоимости объекта недвижимости, специалисту требуется значительное время. Определение рыночной стоимости как наиболее вероятной цены,
которую продавец и покупатель готовы заплатить за объект недвижимости без какого-либо принуждения происходит под
влиянием множества факторов. Автоматизация позволит ускорить процесс принятия решения, учесть
большее количество факторов и снизить уровень субъективности.
Целью исследования является проверка адекватности применения методов эконометрического
анализа для оценки объектов недвижимости и построение на их основе модели стоимости, проектирование нейронной сети для
прогнозирования стоимости недвижимости, сравнение полученных результатов с целью выявления более точной модели и, как
следствие, прогнозирование более точной стоимости. Результаты исследования могут быть полезны для прогнозирования
ценообразования на рынке недвижимости, а также при оценке стоимости объектов недвижимости.
