\subsection{Развертывание программного средства}
\label{sec:design:deployment}

После выявления, а также завершения проектирования всех компонентов программного средства появляется вопрос о
планировании развертывания всей системы. Необходимо составить описание требуемых аппаратно-программных комплексов,
которые понадобятся для обеспечения функционирования распределенного приложения. Для этих целей целесообразным выглядит
составление диаграммы развертывания стандарта UML 2.1.

Данная диаграмма представлена на рисунке~\ref{fig:design:deployment:diagram}. Она отражает следующие особенности развертывания:

\begin{figure}[!ht]
  \centering
    \includegraphics[scale=0.45]{deployment.png}
    \caption{Диаграмма развертывания ПС}
    \label{fig:design:deployment:diagram}
  \end{figure}

\begin{itemize}
  \item на узле оконечного устройства в качестве среды выполнения перечислен список браузерных программных средств, с
  помощью которых можно использовать клиентскую часть приложения;
	\item в качестве операционной системы для сервера клиентской части перечислен список поддерживаемых веб-сервером ОС;
	\item при необходимости Apache HTTP Server может быть заменен другим HTTP-сервером;
  \item для серверной части программного средства и базы данных показано их развертывание на отдельных узлах. При
  самом развертывании в зависимости от условий поставщика вычислительных мощностей данные элементы программной системы
  могут быть объединены на одном узле;
  \item в свою очередь, помимо упрощения, возможно и усложнение схемы развертывания, например, база данных будет
  развернута на нескольких узлах. Тем не менее, все узлы должны удовлетворять отображенным условиям;
  \item все серверы: и клиентской части, и серверной, и базы данных, -- могут быть физически расположены в
  различных дата-центрах;
  \item предполагается, что пользовательское оконечное устройство зна\-чи\-те\-льно удалено от серверов программной
  системы, доступ осуществляется через сеть Интернет, что и упрощенно показано на диаграмме.
\end{itemize}

Таким образом, после развертывания программного средства пользователи могут уже начать им пользоваться.
