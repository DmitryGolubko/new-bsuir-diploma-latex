\subsection{Разработка спецификации функциональных требований}
\label{sec:domain:specification}

С учетом требований, определенных в подразделе \ref{sec:analysis:specification}, представим детализацию функций
проектируемого ПС.

\subsubsection{} Функция регистрации
\label{sec:domain:specification:signup}

Функция регистрации должна быть реализована с учетом следующих требований:

\begin{enumerate}
  \item процесс регистрации инициируется пользователем системы (на рисунке~\ref{fig:domain:model:use_cases}
  представлен в виде роли <<Гость>>);
	\item для регистрации пользователь обязан предоставить адрес электронной почты и установить пароль;
  \item правильность предоставленного адреса электронной почты должна проверяться путем отправки письма со ссылкой,
  переход по которой означает подтверждение пользователя;
  \item хранение пароля допускается только в хешированном виде; применяющийся алгоритм должен по криптостойкости быть
  равным или превосходить алгоритмы семейства SHA-2. Использование соли обязательно;
  \item должна быть предусмотрена возможность смены пароля и после регистрации;
  \item должна быть предусмотрена возможность регистрации по приглашению.
\end{enumerate}

\subsubsection{} Функция аутентификации
\label{sec:domain:specification:authentication}

Функция аутентификации должна быть реализована с учетом следующих требований:

\begin{enumerate}
  \item инициатором является пользователь, при этом ему необходимо предоставить адрес электронной почты и пароль,
  заданные при регистрации;
  \item должна быть реализована возможность повторной аутентификации пользователя без необходимости ввода
  какой-либо информации;
	\item должна быть реализована возможность восстановления пароля:
	\begin{enumerate}
		\item для восстановления пароля пользователь должен предоставить адрес электронной почты, зарегистрированный в системе;
		\item на предоставленный адрес высылается уникальная ссылка;
		\item после перехода пользователем по данной ссылке ему предоставляется возможность установить новый пароль.
	\end{enumerate}
\end{enumerate}

\subsubsection{} Система ролей
\label{sec:domain:specification:roles}

При реализации системы ролей следует учесть требования:

\begin{enumerate}
	\item должны быть реализованы следующие роли:
	\begin{enumerate}
		\item обычный пользователь;
		\item технический менеджер;
		\item менеджер по продажам;
		\item HR менеджер;
		\item администратор;
	\end{enumerate}
	\item для роли администратор должна быть реализована возможность назначения любого пользователя менеджером;
	\item для роли администратор должны быть реализована возможность \linebreak управления списком прав каждого вида менеджеров
	\item должна быть реализована возможность наличия у одного пользователя нескольких разных ролей.
\end{enumerate}

\subsubsection{} Функция заполнения профиля
\label{sec:domain:specification:profile}

Заполнение профиля входит в число основных функций разрабатываемого приложения. При реализации функции заполнения
профиля следует учесть требования:

\begin{enumerate}
	\item необходимо обеспечить внесение всей необходимой информации в удобном и понятном виде;
	\item необходимо обеспечить редактирование уже заполненного профиля;
  \item перед подтверждением правильности внесенных данных профиль\linebreak должен пройти проверку трех менеджеров
  департамента;
	\item Из заполнненого и подтвержденного профиля можно создать копию.
\end{enumerate}

\subsubsection{} Функция генерации резюме
\label{sec:domain:specification:resume}

Генерация резюме также входит в число основных функций разрабатываемого приложения. При реализации генерации
резюме следует учесть требования:

\begin{enumerate}
	\item необходимо генерировать резюме в формате, пригодном для вывода на бумажный носитель;
	\item сгенерированное резюме должно содержать всю необходимую информацию из заполненного профиля пользователя;
  \item генерировать резюме могут только пользователи с ролью технический менеджер, менеджер по продажам,
  HR менеджер и администратор.
\end{enumerate}

\subsubsection{} Функция управления проектами
\label{sec:domain:specification:projects}

При реализации управления проектами следует учесть требования:

\begin{enumerate}
	\item управлять проектами могут только менеджеры и администратор;
	\item при необходимости возможность управления проектами может быть отключена у любого вида менеджеров;
  \item необходимо обеспечить создание, редактирование, удаление проектов, их описания;
  \item необходимо обеспечить выбор технологий, используемых на проекте;
  \item необходимо обеспечить добавление, редактирование, удаление пользователей, работающих на проекте, опиасание их
  роли и зоны ответственности;
  \item необходимо обеспечить автоматическое добавление выбранных технологий в профили пользователей, работающих на
  проекте.
\end{enumerate}

\subsubsection{} Функция управления департаментами
\label{sec:domain:specification:departments}

При реализации управления департаментами следует учесть требования:

\begin{enumerate}
	\item управлять департаментами могут только менеджеры и администратор;
	\item при необходимости возможность управления департаментами может быть отключена у любого вида менеджеров;
  \item необходимо обеспечить создание, редактирование, удаление департаментами, их описания;
  \item необходимо обеспечить выбор менеджеров департамента для дальнейшей проверки заполненных профилей;
  \item необходимо обеспечить добавление, редактирование, удаление пользователей, работающих в департаменте, их позицию
  в департаменте.
\end{enumerate}
