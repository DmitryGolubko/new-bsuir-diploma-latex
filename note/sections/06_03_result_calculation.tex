\subsection{Расчет стоимостной оценки результата}
\label{sec:economics:result_calculation}

В качестве экономического эффекта выступает общая экономия всех видов ресурсов при использовании разработанной системы
относительно того, как процесс менеджмента персонала осуществлялся ранее.

Данные для расчета экономии ресурсов в связи с применением разработанной системы менеджмента персонала отображены в
таблице~\ref{table:economics:result_calculation:initial_data}.

\begin{table}[!ht]
  \caption{Исходные данные}
  \label{table:economics:result_calculation:initial_data}
  \centering
    \begin{tabular}{{
    |>{\raggedright}m{0.45\textwidth} | 
      >{\centering}m{0.15\textwidth} |
      >{\centering}m{0.15\textwidth} |
      >{\centering\arraybackslash}m{0.15\textwidth}|}}
      \hline
        {\begin{center} Наименование показателя \end{center}} & Условное обозначение & Значение в базовом варианте & Значение в новом варианте \\
      \hline
        Капитальные вложения, руб. & $ \text{К}_\text{пр} $ & --- & 7914,91 \\
      \hline
        Среднемесячная заработная плата одного специалиста, руб. & $ \text{З}_\text{см} $ & 1500 & 1500\\
      \hline
        Среднемесячное число рабочих дней & $ \text{Д}_\text{р} $ & 18 & 18 \\
      \hline
        Количество задач, решаемых за год & $ \text{З}_\text{т1} $, $ \text{З}_\text{т2} $ & 50 & 30 \\
      \hline
        Объём работ, выполняемый при решении одной задачи & $ \text{А}_\text{1}$, $ \text{А}_\text{2} $ & 16 & 10 \\
      \hline
        Средняя трудоемкость работ в расчете на 1 задачу, чел.-час на задачу & $ \text{Т}_\text{с1}$, $ \text{Т}_\text{с2} $ & 1 & 2 \\
      \hline
      Количество часов работы в день & $ \text{Т}_\text{ч} $ & 8 & 8 \\
      \hline
    \end{tabular}
\end{table}

Так как ПО разрабатывалось для внутреннего использования ООО «СуматоСофт», то общие капитальные затраты на разработку
соответствуют полной себестоимости данного ПО и составят $7914,91$ руб.

Определим общую годовую экономию затрат при использовании системы менеджмента персонала.

Экономия затрат на заработную плату при использовании нового ПО в расчете на количество задач, выполняемых в год
рассчитывается по формуле:

\begin{equation}
  \text{С}_\text{з} = \frac{\text{З}_\text{зм} \cdot (\text{Т}_\text{с1} \cdot \text{З}_\text{т1} \cdot \text{А}_\text{1} - \text{Т}_\text{с2} \cdot \text{З}_\text{т2} \cdot \text{А}_\text{2})}{\text{Т}_\text{ч} \cdot \text{Д}_\text{р}},
\end{equation}
\begin{explanation}
  где & $ \text{З}_\text{зм} $ & среднемесячная заработная плата одного специалиста, руб.;\\
  & $ \text{З}_\text{т1} $, $ \text{З}_\text{т2} $ & количество задач, решаемых в год в базовом варианте и при
  использовании нового ПО соответственно, задач;\\
  & $ \text{Т}_\text{ч} $ & количество часов работы в день, часов;\\
  & $ \text{Т}_\text{с1} $, $ \text{Т}_\text{с2} $ & средняя трудоемкость работ в расчете на 1 задачу в
  базовом варианте и при использовании нового ПО соответственно, человеко-часов;\\
  & $ \text{А}_\text{1} $, $ \text{А}_\text{2} $ & объем выполняемых работ в расчете на 1 задачу в базовом
  варианте и при использовании нового ПО соответственно, обращений;\\
  & $ \text{Д}_\text{р} $ & среднемесячное количество рабочих дней.
\end{explanation}

\begin{equation}
  \text{С}_\text{з} = \frac{1500 \cdot (1 \cdot 50 \cdot 16 - 2 \cdot 30 \cdot 10)}{8 \cdot 18} = 2083,33 \text{ руб.}
\end{equation}

Экономия с учетом начисления на заработную плату определяется по формуле:

\begin{equation}
  \text{С}_\text{н} = \text{С}_\text{з} \cdot \text{К}_\text{нз},
\end{equation}
\begin{explanation}
  где & $ \text{К}_\text{нз} $ & коэффициент начислений на заработную плату, $ \text{К}_\text{нз} = 1,55 $;.
\end{explanation}

\begin{equation}
  \text{С}_\text{н} = 2083,33 \cdot 1,55 = 3229,16 \text{ руб.}
\end{equation}

Общая годовая экономия в результате сокращения текущих затрат, связанных с использованием нового ПО:

\begin{equation}
  \text{С}_\text{о} = \text{С}_\text{н} \cdot \text{С}_\text{з},
\end{equation}

\begin{equation}
  \text{С}_\text{о} = 3229,16 + 2083,33 = 5312,49 \text{ руб.}
\end{equation}

Для пользователя в качестве экономического эффекта выступает чистая прибыль – дополнительная прибыль, остающаяся в его
распоряжении, которая определяется по формуле:

\begin{equation}
  \Delta\text{П}_\text{ч} = \text{С}_\text{о} - \frac{\text{С}_\text{о} \cdot \text{Н}_\text{п}}{100},
\end{equation}
\begin{explanation}
  где & $ \text{Н}_\text{п} $ & ставка налога на прибыль, $ \text{Н}_\text{п} = 18\% $;.
\end{explanation}

Годовой прирост чистой прибыли от полученной экономии ресурсов при использовании разработанной системы управления
продажами составит:

\begin{equation}
  \Delta\text{П}_\text{ч} = 5312,49 - (5312,49 \cdot 18)/100 = 4356.24 \text{ руб.}
\end{equation}

