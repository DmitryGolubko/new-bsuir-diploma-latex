\sectioncentered*{Заключение}
\addcontentsline{toc}{section}{ЗАКЛЮЧЕНИЕ}

Результатом работы над дипломным проектом стало программное средство менеджмента персонала предприятия, задачей
которого является отслежнивание текущих проектов предприятия, текущей занятости сотрудников на проектах, их способоности,
а также на основе полученных данных автоматически генерировать резюме сотрудника для дальнейшего использования в
отделе маркетинга и продаж. Были проанализированы существующие сервисы генерации резюме, системы управления сотрудникам,
исследованы разные направления и подходы к решению подобного рода задач.

На основании проведенного анализа предметной области были выдвинуты требования к программному средству. В качестве
технологий разработки были выбраны наиболее современные существующие на данный момент средства, широко применяемые в
индустрии.

Разработано программное средство, целевой платформой которого является веб-приложение и которое поддерживает следующие
функции:

\begin{itemize}
	\item создание и управление проектами предприятия;
	\item создание и управление текущих позиций и департаментов предприятия;
	\item заполнение и проверка профиля пользователя;
	\item создание копий профиля пользователя с возможностью их редактирования без изменения базового профиля;
	\item генерация резюме пользоателя на основе заполненного профиля;
	\item возможность управления правами доступа для разграничения функций между менеджерами;
	\item возможность приглашения незарегестрированных пользователей в систему.
\end{itemize}

Согласно объявленным требованиям был разработан набор тест-кейсов, которые были успешно пройдены в ходе тестовых
испытаний программного средства. Успешность прохождения тестов показывает корректность работы программы и соответствие
функциональным требованиям.

Подробно описана методика использования программного средства, которая позволяет за достаточно короткие сроки освоить
работу с программой.

Также в ходе работы над дипломным проектом рассмотрена экономическая сторона проектирования и разработки программного
средства, рассчитан экономический эффект от использования программного средства у пользователя. В результате расчётов
подтвердилась целесообразность разработки. Так как разработанная система предоставляет новый способ генерации резюме
сотрудников и использует современные технологии создания программных средств, рентабельность её разработки
составила 42\%, а инвестиции, вложенные в разработку, окупаются на второй год использования программного средства.

Основной целью разработки программы было упрощение процессов генерации резюме сотрудников, отслеживания текущих проектов
и занятости сотрудниках на них. В ходе работы над дипломным проектом эта цель была успешно достигнута.

В будущем планируется добавить возможность обмена сообщениями между сотрудниками, систему учета рабочего времени,
более детальное \linebreak управление проектами, его текущие задачи, их описание и статус. Для генерации резюме добавить возможность
выбора информации для генерации, а также реализовать конструктор шаблонов.
