
\sectioncentered*{Заключение}
\addcontentsline{toc}{section}{ЗАКЛЮЧЕНИЕ}


В ходе работы над магистерской диссертацией были изучены и проанализированы существующие способы и алгоритмы анализа
данных. Были собраны данные о продающихся объектах недвижимости на рынке г. Минска на момент конца 2020 года.
Всего было собрано данных о более чем 8 тысяч проданных объектов недвижимости. После был выполнен регрессионный анализ
полученных данных и анализ с использованием нейронных сетей. Было проведено сравнение полученных результатов и сделаны
соответствующие выводы об их эффективности и возможности реального использования.

На основании проведенных исследований можно утверждать, что применение нейронных сетей для прогнозирования
стоимости объектов недвижимости более эффективно использования методов регрессионного анализа и может достаточно
точно отражать рыночную стоимость недвижимости. Исходя из сравнения полученных результатов нейронных сетей можно сделать
вывод, что наилучшие результаты показывают отдельные
модели по району. Однако результаты варьируются от района к району в зависимости от количества продаваемых объектов
в данном районе, поэтому для отдельных районов целесообразнее использовать модель по районам, для остальных районов - 
общую модель.
Предложенные методы могут быть использованы продавцами для первичной оценки стоимости жилой недвижимости,
а покупателями могут использоваться в качестве дополнительного источника информации. Данные модели можно улучшить
путем сбора данных о проданных объектах недвижимости, а не только продаваемых в текущий момент времени, это увеличит
объем выборки и повысит качество получаемых результатов.
