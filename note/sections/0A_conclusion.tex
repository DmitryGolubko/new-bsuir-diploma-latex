
\sectioncentered*{Заключение}
\addcontentsline{toc}{section}{ЗАКЛЮЧЕНИЕ}


В ходе работы над магистерской диссертацией были изучены и проанализированы существующие способы и алгоритмы анализа
данных. Были собраны данные об проданных объектах недвижимости на рынке за период с 2017 по 2020 год.
Всего было собрано данных о более чем 5 тысяч проданных объектов недвижимости. После был выполнен регресионный анализ
полученных данных и анализ с использованием нейронных сетей. Было проведено сравнение полученных результатов и сделаны
соответствующие выводы об их эффективности и возможности реального использования.

На основании проведенных исследований можно утверждать, что применение нейронных сетей для прогнозирования
стоимости объектов недвижимости более эффективно использования методов регрессионного анализа и может достаточно
точно отражать рыночную стоимость недвижимости. Однако с увеличением числа выборки точность прогнозирования падает ввиду
присутствия множества скрытых факторов(такие как время постройки дома, удаленность от различных сервисов, наличие или
отсутствие мебели, качество постройки дома), не учитываемых в данном исследовании.
Предложенные методы могут быть использованы продавцами для первичной оценки стоимости жилой недвижимости,
а покупателями могут использоваться в качестве дополнительного источника информации, однако данная модель требует
доработки.
