\sectioncentered*{Реферат}
\thispagestyle{empty}

% Зачем: чтобы можно было вывести общее число страниц.
% Добавляется единица, поскольку последняя страница -- ведомость.
\FPeval{\totalpages}{round(\getpagerefnumber{LastPage} + 1, 0)}

\begin{center}
	Пояснительная записка \num{\totalpages}~с., \num{\totfig{}}~рис., \num{\tottab{}}~табл., \num{\toteq{}}~формул, \num{\totref{}}~источников.
	\MakeUppercase{Программное средство менеджмента персонала предприятия с использованием технологии Ruby on Rails}: 
	дипломный проект / Д.В. Голубко --- Минск: БГУИР, 2019.
\end{center}

Объектом исследования и разработки является программное средство анализа программного кода и деятельности разработчика.

Цель настоящего дипломного проекта состоит в разработке программной системы, предназначенной для эффективной
автоматизации процесса менеджмента персонала и генерации резюме. 

При разработке и внедрении приложения используется следующий стек технологий: HTML5, CSS3, JavaScript, React.js,
Ruby, Ruby on Rails, PostgreSQL.

В первом разделе проводится обзор существующих программных средств, менеджмента персонала, генерации резюме,
выделяются положительные и отрицательные особенности существующих аналогов.
Выдвигаются общие требования к созданию программного средства.

Во втором разделе проводится моделирование программного средства, а также разработка функциональных требований.

Третий раздел посвящён разработке архитектуры программного средства, а также модели базы данных.
Также третий раздел описывает необходимые технологии для разработки программного средства и процесс разработки ПС.

В четвертом разделе содержится информация о тестировании разработанного приложения на соответствие функциональным требованиям.

Пятый раздел содержит руководство пользователя.

В шестом разделе приведено технико-экономическое обоснование разработки и внедрения программного средства.

Заключение содержит краткие выводы по дипломному проекту.

Дипломный проект является завершённым, поставленная задача решена в полной мере. Планируется дальнейшее развитие
программного средства и расширение его функциональности. Проект выполнен самостоятельно, проведён анализ оригинальности
в системе «Антиплагиат». Процент оригинальности составляет 81,92\%. Цитирования обозначены ссылками на публикации,
указанные в «Списке использованных источников».
